Los casos de error posibles son todas aquellas transiciones no recogidas por nuestro AFD. En estos casos, se generará un error léxico y se informará al usuario de la existencia de un error en el código fuente, junto con el número de línea y columna en la que se ha detectado el mismo. A continuación, detallamos los errores que hemos identificado y los mensajes de error correspondientes a cada uno:\\
\begin{itemize}[left=1cm]
	\item \textbf{Caracteres no reconocidos.} En caso de que el carácter leído no coincida con ninguna transición del AFD, de manera que no pueda transitar desde el estado 0, se generará un error léxico con el siguiente mensaje: \textcolor{red}{(línea:columna) ERROR: Carácter inesperado al buscar el siguiente símbolo («carácter inesperado», U+{:04X}).}
    \item \textbf{Comentario de bloque mal abierto.} En caso de que tras encontrar una `/' no se encuentre un «*», se generará un error léxico con el siguiente mensaje: \textcolor{red}{(línea:columna) ERROR: Carácter inesperado tras «/». Se esperaba «*» para abrir un comentario de bloque}
    \item \textbf{Comentario de bloque no cerrado.} Si se detecta un comentario de bloque sin cerrar, se generará un error léxico con el siguiente mensaje: \textcolor{red}{(línea:columna) ERROR: Fin de fichero inesperado. Se esperaba «*/» para cerrar el comentario de bloque.}
	\item \textbf{Valor entero fuera de rango.} Si el valor de una constante entera supera el rango permitido que abarcan los enteros de 16 bits con signo, se generará un error léxico con el siguiente mensaje: \textcolor{red}{(línea:columna) ERROR: El valor del entero es demasiado grande (máximo 32767).}
    \item \textbf{Cadena de caracteres no cerrada.} Si se detecta una cadena de caracteres sin cerrar, se generará un error léxico con el siguiente mensaje: \textcolor{red}{(línea:columna) ERROR: Fin de fichero inesperado. Se esperaba «'» para cerrar la cadena.}
    \item \textbf{Secuencia de escape no válida.} En este caso, distinguiremos dos situaciones:
    \begin{itemize}
        \item \textbf{Carácter imprimible tras la barra invertida.} Si el carácter siguiente a la barra invertida es un carácter imprimible, pero no forma ninguna secuencia de escape válida, se generará un error léxico con el siguiente mensaje: \textcolor{red}{(línea:columna) ERROR: Error en la cadena, la secuencia de escape «\textbackslash carácter» (U+{:04X}) no es válida.}
        \item \textbf{Carácter ilegal tras la barra invertida.} Si el carácter que sigue a la barra invertida no es un carácter imprimible, se generará un error léxico con el siguiente mensaje: \textcolor{red}{(línea:columna) ERROR: Error en la cadena, carácter ilegal en la secuencia de escape (U+{:04X}).}
    \end{itemize}
    \item \textbf{Carácter no permitido en la cadena.} Si se detecta un carácter no permitido en la cadena, como un salto de línea, se generará un error léxico con el siguiente mensaje: \textcolor{red}{(línea:columna) ERROR: Error en la cadena, carácter no permitido (U+{:04X}).}
    \item \textbf{Cadena de caracteres demasiado larga.} Si la longitud de la cadena supera los 64 caracteres, se generará un error léxico con el siguiente mensaje (donde x es el número de caracteres de la cadena): \textcolor{red}{(línea:columna) ERROR: La longitud de cadena excede el límite de 64 caracteres (x caracteres).}
	\item \textbf{Operadores lógicos incorrectos.} En caso de haber un \verb!&! en lugar de \verb!&&! o un \verb!|! en lugar de \verb!||!, se generará un error léxico con el siguiente mensaje (donde op será \verb!&! o \verb!|!): \textcolor{red}{(línea:columna) ERROR: Se esperaba «op» después de «op» para formar un operador.}
\end{itemize}
