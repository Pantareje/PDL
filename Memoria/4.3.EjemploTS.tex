\subsection*{1. Código de Ejemplo}
\begin{verbatim}
int x;               // Variable global
void func1() {
    int a;           // Variable local a func1
    x = a + 10;      // Uso de variable global y local
}
void func2() {
    int x;           // Variable local a func2 (oculta a la global)
    x = 20;          // Uso de variable local de func2
}
\end{verbatim}

\subsection*{2. Estructura de las Tablas de Símbolos}

\paragraph{Tabla Global:} Contiene símbolos con alcance global.

\vspace{1mm}

\begin{tabular}{|c|c|c|}
\hline
\textbf{Símbolo} & \textbf{Tipo} & \textbf{Información Adicional} \\
\hline
x       & int  & Variable global                \\
func1   & void & Función (sin retorno)          \\
func2   & void & Función (sin retorno)          \\
\hline
\end{tabular}

\paragraph{Tabla Local de func1:} Contiene los símbolos declarados en func1.

\vspace{1mm}

\begin{tabular}{|c|c|c|}
\hline
\textbf{Símbolo} & \textbf{Tipo} & \textbf{Información Adicional} \\
\hline
a       & int  & Variable local                 \\
\hline
\end{tabular}

\paragraph{Tabla Local de func2:} Contiene los símbolos declarados en func2.

\vspace{1mm}

\begin{tabular}{|c|c|c|}
\hline
\textbf{Símbolo} & \textbf{Tipo} & \textbf{Información Adicional} \\
\hline
x       & int  & Variable local (oculta a la global) \\
\hline
\end{tabular}
