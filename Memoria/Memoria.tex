\documentclass{article}

% Soporte para Unicode.
% En vez de usar pdflatex, usa el comando xelatex.

\usepackage{fontspec}
\setmainfont{Latin Modern Roman}
\setmonofont{JetBrains Mono}

\usepackage[spanish]{babel} % Configuración para español
\usepackage[margin=1in]{geometry} % márgenes de la página
\usepackage{enumitem} % para listas personalizadas
\usepackage{fancyhdr} % para encabezados y pies de página
\usepackage[table]{xcolor} % para colores en tablas
\usepackage{xcolor} % para colores personalizados
\usepackage{hyperref} % para enlaces
\usepackage{tikz} % para dibujar diagramas (AFD)
\usepackage[most]{tcolorbox} % para cajas de texto
\usepackage{indentfirst} % para la sangría al iniciar una sección
\usepackage{listings} % para los códigos del anexo

\setlength{\headheight}{13.07225pt}
\addtolength{\topmargin}{-1.07225pt}

\pagestyle{fancy}
\fancyhead[L]{Grupo 7}
\fancyhead[C]{}
\fancyfoot[L]{Práctica 2 - Procesadores de Lenguajes}
\fancyfoot[C]{}
\fancyfoot[R]{\thepage}

\title{\textbf{Práctica 2 - Procesadores de Lenguajes}}
\author{\textbf{Grupo 7}\\Carmen Toribio Pérez, 22M009\\Sergio Gil Atienza, 22M046\\María Moronta Carrión, 22M111}
\date{}

\begin{document}

\maketitle

\tableofcontents

\section*{Introducción}

El proyecto a continuación consiste en la construcción de un procesador diseñado para analizar y verificar la corrección léxica, sintáctica y semántica del lenguaje JS--.

El trabajo comenzó con el análisis del lenguaje fuente para reconocer sus elementos fundamentales, lo que nos permitió identificar sus tokens. Gracias a ello, pudimos crear la gramática del lenguaje y su Autómata Finito Determinista equivalente. Con todo esto logramos implementar el Analizador Léxico, así como crear un diseño inicial de la Tabla de Símbolos, un núcleo fundamental para la organización de la información durante el análisis. 

A continuación, desarrollamos un Analizador Sintáctico, gracias a la identificación de una nueva gramática que nos permitió corregir la estructura del lenguaje. Finalmente, a esta gramática se integraron las Acciones Semánticas, permitiéndonos realizar la Traducción Dirigida por la Sintaxis propia del Analizador Semántico. 

Todo esto fue complementado por un Gestor de Errores claro y conciso, que mejora la experiencia de usuario a la hora de encontrar fallos localizados. El resultado final es un procesador capaz de interpretar un programa escrito en JS--. Esto queda demostrado en los distintos casos de prueba que incluimos. 

Como integrantes del \textbf{grupo 7}, hemos tenido que cumplir con las siguientes especificaciones: 
\begin{itemize}[left=2cm]
    \item Sentencias: Sentencia repetitiva (for)
    \item Operadores especiales: Asignación con suma (+=)
    \item Técnicas de Análisis Sintáctico: Descendente Recursivo
    \item Comentarios: Comentario de bloque (/* */)
    \item Cadenas: Con comillas simples (' ')
\end{itemize}

Además, hemos decidido usar \textbf{C++ como lenguaje de programación} ya que la mayor parte de la infraestructura de compiladores está escrita en C o en C++, incluyendo MSVC (desarrollado por Microsoft) y el proyecto LLVM (utilizado por Google en Android). También hemos tenido en cuenta su flexibilidad y potencia, junto a la amplia variedad de utilidades que tiene su librería estándar. En comparación con lenguajes como Java o JavaScript, C++ ofrece mayor eficiencia y control sobre los recursos del sistema, a la vez que se mantiene versátil y permite escribir código legible.

\newpage

\section{Analizador Léxico}
El \textbf{Analizador Léxico} constituye la primera etapa en el procesamiento de un lenguaje, pues es el encargado de interactuar directamente con el fichero fuente. Su propósito principal es identificar y clasificar las unidades mínimas del lenguaje, conocidas como \textbf{tokens}. Por ello, el primer paso es su identificación.

\subsection{Tokens}
Para hacer la lista de tokens nos hemos basado en la actividad práctica de la plataforma Draco. Hemos decidido utilizar el mismo formato en tablas con tal de facilitar su legibilidad.\\

{\small
Tokens obligatorios

\begin{table}[h!]
    \centering
    \small
    \begin{tabular}{|l|l|l|}
        \hline
		\rowcolor{gray!20} % color grisáceo en la cabecera de la tabla
        \textbf{Elemento} & \textbf{Código de Token} & \textbf{Atributo} \\ \hline
        boolean & bool & - \\ \hline
        for & for & - \\ \hline
        function & fn & - \\ \hline
        if & if & - \\ \hline
        input & in & - \\ \hline
        int & int & - \\ \hline
        output & out & - \\ \hline
        return & ret & - \\ \hline
        string & str & - \\ \hline
        var & var & - \\ \hline
        void & void & - \\ \hline
        constante entera & cint & Número \\ \hline
        Cadena (') & cstr & Cadena ("c*") \\ \hline
		Identificador & id & Número (posición en la TS) \\ \hline
        += & cumass & - \\ \hline
        = & ass & - \\ \hline
        , & com & - \\ \hline
        ; & scol & - \\ \hline
        ( & po & - \\ \hline
        ) & pc & - \\ \hline
        \{ & cbo & - \\ \hline
        \} & cbc & - \\ \hline
    \end{tabular}
\end{table}

Tokens de operadores aritméticos, lógicos y relacionales:
\begin{table}[h!]
    \centering
    \small
    \begin{tabular}{|l|l|l|}
        \hline
		\rowcolor{gray!20} % color grisaceo en la cabecera de la tabla
        \textbf{Grupo de Opciones} & \textbf{Código de Token} & \textbf{Atributo} \\ \hline
        Grupo Operadores Aritméticos: Suma (+) & sum & - \\ \hline
        Grupo Operadores Aritméticos: Resta (-) & sub & - \\ \hline
        Grupo Operadores Lógicos: Y lógico (\&\&) & and & - \\ \hline
        Grupo Operadores Lógicos: O lógico (\texttt{||}) & or & - \\ \hline
        Grupo Operadores Relacionales: Menor (\textless) & ls & - \\ \hline
        Grupo Operadores Relacionales: Mayor (\textgreater) & gr & - \\ \hline
    \end{tabular}
\end{table}

Tokens opcionales:
\begin{table}[h!]
    \centering
    \small
    \begin{tabular}{|l|l|l|}
        \hline
		\rowcolor{gray!20} % color grisáceo en la cabecera de la tabla
        \textbf{Grupo de Opciones} & \textbf{Código de Token} & \textbf{Atributo} \\ \hline
        %Estas dos las quitamos al final
        %Menos Unario (-) & sub & - \\ \hline
        %Más Unario (+) & sum & - \\ \hline
        false & cap & - \\ \hline
        true & nocap & - \\ \hline
        EOF & eof & - \\ \hline
    \end{tabular}
\end{table}}

Por tanto, los siguientes tipos de expresiones no serán identificados como tokens: los delimitadores (como los espacios en blanco o las tabulaciones), los comentarios de bloque (/* */) o los saltos de línea (\textbackslash n).

\subsection{Gramática Regular del Analizador Léxico}

En esta sección, describimos la Gramática Regular (gramática de tipo 3 según la jerarquía de Chomsky) que hemos diseñado para identificar y generar los tokens del lenguaje fuente. 

\begin{center}
    \begin{tcolorbox}[title=Símbolos no terminales, width=0.57\textwidth]
        $d$ := 0...9
        
        $l$ := a...z, A...Z
        
        $c_1$ := espacio o cualquier carácter imprimible menos \textbackslash
        
        $cesc$ := carácter escapable (', 0, n, a, t, v, f, r, \textbackslash )
        
        $c_2$ := cualquier carácter menos * y eof
        
        $c_3$ := cualquier carácter menos *, / y eof
    \end{tcolorbox}
\end{center}

\begin{tcolorbox}[title=Gramática Regular]
    \hspace{0.5cm} S → del S \texttt{|} $l$A \texttt{|} $d$B \texttt{|} 'C \texttt{|} +D\texttt{|} - \texttt{|} = \texttt{|} \(>\) \texttt{|} \(<\) \texttt{|} \&E \texttt{|} \texttt{|}F \texttt{|} /G \texttt{|} \} \texttt{|} \{ \texttt{|} ) \texttt{|} ( \texttt{|} ; \texttt{|} , \texttt{|} eof
    
    \hspace{0.5cm} A → $l$A \texttt{|} $d$A \texttt{|} \_A \texttt{|} $\lambda$
    
    \hspace{0.5cm} B → $d$B \texttt{|} $\lambda$
    
    \hspace{0.5cm} C → $c_1$C \texttt{|} \textbackslash C' \texttt{|} '
    
    \hspace{0.5cm} C' → $cesc$C
    
    \hspace{0.5cm} D → = \texttt{|} $\lambda$
    
    \hspace{0.5cm} E → \&
    
    \hspace{0.5cm} F → \texttt{|}
    
    \hspace{0.5cm} G → *H
    
    \hspace{0.5cm} H → $c_2$H \texttt{|} *I
    
    \hspace{0.5cm} I → /S \texttt{|} $c_3$H \texttt{|} *I
\end{tcolorbox}


\subsection{Autómata Finito Determinista}
Una vez definida la gramática regular, el siguiente paso es construir el Autómata Finito Determinista (AFD) correspondiente. A continuación, presentamos su diseño, incluyendo las transiciones entre estados.

\begin{center}
\begin{tikzpicture}[scale=0.15]
\tikzstyle{every node}+=[inner sep=0pt]
\draw [black] (38.4,-29.3) circle (3);
\draw (38.4,-29.3) node {$0$};
\draw [black] (7.3,-18.4) circle (3);
\draw (7.3,-18.4) node {$1$};
\draw [black] (6.2,-5.8) circle (3);
\draw (6.2,-5.8) node {$2$};
\draw [black] (6.2,-5.8) circle (2.4);
\draw [black] (15.1,-13.4) circle (3);
\draw (15.1,-13.4) node {$3$};
\draw [black] (13.9,-3.7) circle (3);
\draw (13.9,-3.7) node {$4$};
\draw [black] (13.9,-3.7) circle (2.4);
\draw [black] (31.2,-14.6) circle (3);
\draw (31.2,-14.6) node {$5$};
\draw [black] (31.2,-3.7) circle (3);
\draw (31.2,-3.7) node {$6$};
\draw [black] (42.8,-6.5) circle (3);
\draw (42.8,-6.5) node {$7$};
\draw [black] (42.8,-6.5) circle (2.4);
\draw [black] (49.3,-13.4) circle (3);
\draw (49.3,-13.4) node {$8$};
\draw [black] (66.5,-9.8) circle (3);
\draw (66.5,-9.8) node {$9$};
\draw [black] (66.5,-9.8) circle (2.4);
\draw [black] (73.3,-26.4) circle (3);
\draw (73.3,-26.4) node {$10$};
\draw [black] (73.3,-26.4) circle (2.4);
\draw [black] (57.8,-38.9) circle (3);
\draw (57.8,-38.9) node {$11$};
\draw [black] (70.8,-35.9) circle (3);
\draw (70.8,-35.9) node {$12$};
\draw [black] (70.8,-35.9) circle (2.4);
\draw [black] (56.9,-47.8) circle (3);
\draw (56.9,-47.8) node {$13$};
\draw [black] (67.5,-48.5) circle (3);
\draw (67.5,-48.5) node {$14$};
\draw [black] (67.5,-48.5) circle (2.4);
\draw [black] (18.3,-43.6) circle (3);
\draw (18.3,-43.6) node {$15$};
\draw [black] (13.9,-51.8) circle (3);
\draw (13.9,-51.8) node {$16$};
\draw [black] (4.7,-47.2) circle (3);
\draw (4.7,-47.2) node {$17$};
\draw [black] (29.6,-55.5) circle (3);
\draw (29.6,-55.5) node {$18$};
\draw [black] (29.6,-55.5) circle (2.4);
\draw [black] (12.6,-31.6) circle (3);
\draw (12.6,-31.6) node {$19$};
\draw [black] (12.6,-31.6) circle (2.4);
\draw [black] (35.41,-29.054) arc (-96.15072:-122.4786:59.606);
\fill [black] (9.79,-20.07) -- (10.2,-20.93) -- (10.73,-20.08);
\draw (21.34,-26.57) node [below] {$l$};
\draw [black] (36.458,-27.028) arc (248.25582:-39.74418:2.25);
\draw (35.31,-22.12) node [above] {$del$};
\fill [black] (39.02,-26.38) -- (39.78,-25.82) -- (38.85,-25.45);
\draw [black] (42.9,-21.9) -- (39.96,-26.74);
\fill [black] (39.96,-26.74) -- (40.8,-26.31) -- (39.95,-25.79);
\draw [black] (7.04,-15.41) -- (6.46,-8.79);
\fill [black] (6.46,-8.79) -- (6.03,-9.63) -- (7.03,-9.54);
\draw (6.12,-12.17) node [left] {$o.c.$};
\draw [black] (7.94,-21.319) arc (40.10787:-247.89213:2.25);
\draw (3.61,-25.66) node [below] {$l,\mbox{ }d,\mbox{ }\_$};
\fill [black] (5.37,-20.68) -- (4.44,-20.82) -- (5.08,-21.58);
\draw [black] (35.659,-28.081) arc (-115.2029:-133.41663:70.465);
\fill [black] (17.23,-15.51) -- (17.47,-16.42) -- (18.16,-15.69);
\draw (24.83,-23.03) node [below] {$d$};
\draw [black] (17.112,-11.19) arc (165.42106:-122.57894:2.25);
\draw (22.13,-10.07) node [right] {$d$};
\fill [black] (18.08,-13.65) -- (18.73,-14.34) -- (18.98,-13.37);
\draw [black] (14.73,-10.42) -- (14.27,-6.68);
\fill [black] (14.27,-6.68) -- (13.87,-7.53) -- (14.86,-7.41);
\draw (15.17,-8.42) node [right] {$o.c.$};
\draw [black] (36.047,-27.446) arc (-133.65734:-174.15181:15.848);
\fill [black] (31.22,-17.6) -- (30.81,-18.44) -- (31.8,-18.34);
\draw (32.05,-24.04) node [left] {$'$};
\draw [black] (28.864,-16.463) arc (-23.69198:-311.69198:2.25);
\draw (23.86,-16.4) node [left] {$o.c.$};
\fill [black] (28.3,-13.88) -- (27.77,-13.1) -- (27.37,-14.02);
\draw [black] (28.826,-12.819) arc (-141.47856:-218.52144:5.89);
\fill [black] (28.83,-5.48) -- (27.94,-5.8) -- (28.72,-6.42);
\draw (27.04,-9.15) node [left] {$\backslash$};
\draw [black] (31.2,-6.7) -- (31.2,-11.6);
\fill [black] (31.2,-11.6) -- (31.7,-10.8) -- (30.7,-10.8);
\draw (31.7,-9.15) node [right] {$cesc$};
\draw [black] (41.24,-9.056) arc (-37.41152:-72.73715:14.277);
\fill [black] (41.24,-9.06) -- (40.36,-9.39) -- (41.15,-10);
\draw (38.8,-12.59) node [below] {$'$};
\draw [black] (48.233,-16.203) arc (-23.31516:-45.54881:34.871);
\fill [black] (48.23,-16.2) -- (47.46,-16.74) -- (48.38,-17.14);
\draw (45.57,-23.47) node [right] {$+$};
\draw [black] (51.777,-11.713) arc (119.84259:83.80039:19.447);
\fill [black] (63.55,-9.25) -- (62.81,-8.66) -- (62.7,-9.66);
\draw (56.84,-8.96) node [above] {$=$};
\draw [black] (63.934,-11.35) arc (-62.70739:-93.64964:22.333);
\fill [black] (63.93,-11.35) -- (62.99,-11.27) -- (63.45,-12.16);
\draw (59.5,-14.04) node [below] {$o.c.$};
\draw [black] (40.965,-27.747) arc (118.831:70.66911:36.333);
\fill [black] (70.51,-25.29) -- (69.92,-24.55) -- (69.59,-25.5);
\draw (55.37,-22.81) node [above] {$-$};
\draw [black] (54.811,-38.667) arc (-97.84966:-134.80692:25.367);
\fill [black] (54.81,-38.67) -- (54.09,-38.06) -- (53.95,-39.05);
\draw (45.8,-36.78) node [below] {$\&$};
\draw [black] (68.375,-37.656) arc (-60.37001:-93.64076:13.669);
\fill [black] (68.38,-37.66) -- (67.43,-37.62) -- (67.93,-38.49);
\draw (65.51,-39.69) node [below] {$\&$};
\draw [black] (54.207,-46.479) arc (-118.50722:-151.49278:36.082);
\fill [black] (54.21,-46.48) -- (53.74,-45.66) -- (53.27,-46.54);
\draw (44.83,-40.76) node [below] {$|$};
\draw [black] (64.579,-49.159) arc (-83.54324:-104.01316:13.736);
\fill [black] (64.58,-49.16) -- (63.73,-48.75) -- (63.84,-49.75);
\draw (62.04,-49.78) node [below] {$|$};
\draw [black] (20.506,-41.567) arc (131.66584:119.19351:86.147);
\fill [black] (20.51,-41.57) -- (21.44,-41.41) -- (20.77,-40.66);
\draw (27.04,-35.23) node [above] {$/$};
\draw [black] (6.881,-45.141) arc (132.06374:103.88692:66.499);
\fill [black] (35.47,-29.95) -- (34.58,-29.66) -- (34.82,-30.63);
\draw (19.45,-35.28) node [above] {$/$};
\draw [black] (19.291,-46.387) arc (2.91612:-59.35083:5.157);
\fill [black] (16.77,-51.09) -- (17.71,-51.11) -- (17.2,-50.25);
\draw (19.36,-50.26) node [right] {$*$};
\draw [black] (10.918,-51.615) arc (-102.36433:-130.76577:9.753);
\fill [black] (6.64,-49.47) -- (6.92,-50.38) -- (7.57,-49.62);
\draw (7.65,-51.32) node [below] {$*$};
\draw [black] (7.684,-47.339) arc (78.23355:48.63635:9.442);
\fill [black] (12,-49.5) -- (11.73,-48.59) -- (11.07,-49.34);
\draw (12.03,-47.63) node [above] {$o.c.$};
\draw [black] (2.839,-44.862) arc (246.26123:-41.73877:2.25);
\draw (2.65,-40.02) node [above] {$*$};
\fill [black] (5.42,-44.3) -- (6.2,-43.77) -- (5.29,-43.37);
\draw [black] (35.41,-29.57) -- (15.59,-31.33);
\fill [black] (15.59,-31.33) -- (16.43,-31.76) -- (16.34,-30.76);
\draw (25.21,-29.8) node [above] {$eof$};
\draw [black] (28.368,-52.768) arc (-160.31098:-236.82114:18.77);
\fill [black] (28.37,-52.77) -- (28.57,-51.85) -- (27.63,-52.18);
\draw (27.48,-39.76) node [left] {$,$};
\draw [black] (29.231,-52.524) arc (-175.91406:-221.21806:28.862);
\fill [black] (29.23,-52.52) -- (29.67,-51.69) -- (28.68,-51.76);
\draw (29.89,-40.57) node [left] {$;$};
\draw [black] (30.052,-52.534) arc (170.14119:152.72669:71.764);
\fill [black] (30.05,-52.53) -- (30.68,-51.83) -- (29.7,-51.66);
\draw (31.96,-41.26) node [left] {$($};
\draw [black] (37.44,-32.14) -- (30.56,-52.66);
\fill [black] (30.56,-52.66) -- (31.28,-52.06) -- (30.34,-51.74);
\draw (34.77,-43.11) node [right] {$)$};
\draw [black] (38.432,-32.299) arc (-1.66748:-35.46464:37.795);
\fill [black] (31.44,-53.13) -- (32.31,-52.77) -- (31.49,-52.19);
\draw (37.25,-43.94) node [right] {$\{$};
\draw [black] (39.338,-32.147) arc (14.16948:-51.3016:21.117);
\fill [black] (32.07,-53.8) -- (33,-53.69) -- (32.38,-52.91);
\draw (39.65,-44.75) node [right] {$\}$};
\draw [black] (41.285,-28.477) arc (104.87742:84.62269:82.844);
\fill [black] (70.32,-26.06) -- (69.57,-25.49) -- (69.48,-26.49);
\draw (55.55,-25.41) node [above] {$=$};
\draw [black] (70.686,-27.871) arc (-62.84962:-107.65027:38.794);
\fill [black] (70.69,-27.87) -- (69.75,-27.79) -- (70.2,-28.68);
\draw (56.34,-32.59) node [below] {$<$};
\draw [black] (70.362,-27.007) arc (-78.97548:-91.52441:132.966);
\fill [black] (70.36,-27.01) -- (69.48,-26.67) -- (69.67,-27.65);
\draw (56.09,-29.58) node [below] {$>$};
\draw [black] (16.693,-52.862) arc (96.92389:-191.07611:2.25);
\draw (20.58,-57.23) node [below] {$o.c.$};
\fill [black] (14.76,-54.66) -- (14.36,-55.52) -- (15.35,-55.4);
\end{tikzpicture}
\end{center}

\subsection{Acciones semánticas}
A lo largo de esta sección, describimos las acciones semánticas que hemos añadido a las transiciones del Autómata Finito Determinista. Estas acciones permiten que se lleven a cabo acciones como la lectura, la identificación de los tokens o, en su lugar, una correcta gestión de los errores léxicos. A continuación, detallamos las operaciones utilizadas y, por legibilidad en el AFD, las transiciones en las que se realiza cada una:\\
\begin{tabbing}
    \hspace{1cm}\=\hspace{5cm}\=\kill
    \textbf{Leer:} \\
    \> car := leer(); \> // Todas las transiciones salvo 1:2, 3:4, 6:5, 8:9 con un carácter\\
    \> \hspace{5cm}// distinto de '=', y 0:19\\
    \> car := leerCesc(); \> // Transición 6:5. Leído un caracter de escape, devuelve el carácter\\
    \> \hspace{5cm}// correspondiente (por ejemplo un 'n' devuelve un eol)
    \\
    \textbf{Concatenar:} \\
    \> lex := car; \> // Transición 0:1 \\
    \> lex := $\emptyset$; \> // Transición 0:5 \\
    \> lex := lex $\oplus$  car; \> // Transiciones 1:1, 5:5 y 6:5\\
    \\
    \textbf{Contador:} \\
    \> // Utilizado para contar el número de caracteres de la cadena \\
    \> cont := 0; \> // Transición 0:5. Inicializa el contador a 0\\
    \> cont := cont + 1; \> // Transiciones 5:5 y 6:5 \\
    \\
    \textbf{Calcular valor entero:} \\
    \> // La función val(car) devuelve el valor entero del dígito correspondiente al carácter car\\
    \> num := val(car); \> // Transición 0:3 \\
    \> num := num * 10 + val(car); \> // Transición 3:3 \\
    \\
    \hspace{1cm}\=\hspace{1cm}\=\hspace{7cm}\=\kill %cambio la indentación porque ahora habrá más tabulaciones que hacer
    \textbf{Generar token:} \\
    \textit{Cadenas y enteros} \\
    \> G1 \> if (cont $>$ 64) \> // Transición 5:7 \\
    \> \hspace{1.5cm} then error (COD\_ERROR\_STRLEN, lex) \\
    \> \hspace{1cm}else Gen\_token(cstr, lex) \\
    \> G2 \> if (num $>$ 32767) \> // Transición 3:4 \\
    \> \hspace{1.5cm} then error (COD\_ERROR\_MAXINT, num) \\
    \> \hspace{1cm}else Gen\_token(cint, num) \\
    \textit{Delimitadores y operadores de control} \\
    \> G3 \> Gen\_token(cumass, -) \> // Transición 8:9 con el carácter '='\\
    \> G4 \> Gen\_token(ass, -) \> // Transición 0:10 con el carácter '='\\
    \> G5 \> Gen\_token(com, -) \> // Transición 0:18 con el carácter ','\\
    \> G6 \> Gen\_token(scol, -) \> // Transición 0:18 con el carácter ';'\\
    \> G7 \> Gen\_token(po, -) \> // Transición 0:18 con el carácter '('\\
    \> G8 \> Gen\_token(pc, -) \> // Transición 0:18 con el carácter ')'\\
    \> G9 \> Gen\_token(cbo, -) \> // Transición 0:18 con el carácter '\{'\\
    \> G10 \> Gen\_token(cbc, -) \> // Transición 0:18 con el carácter '\}' \\
    \textit{Operadores Aritméticos, Lógicos y Relacionales} \\
    \> G11 \> Gen\_token(sum, -) \> // Transición 8:9 con un carácter distinto de '='\\
    \> G12 \> Gen\_token(sub, -) \> // Transición 0:10 con el carácter '-'\\
    \> G13 \> Gen\_token(and, -) \> // Transición 11:12 \\
    \> G14 \> Gen\_token(or, -) \> // Transición 13:14 \\
    \> G15 \> Gen\_token(ls, -) \> // Transición 0:10 con el carácter '$<$'\\
    \> G16 \> Gen\_token(gr, -) \> // Transición 0:10 con el carácter '$>$'\\
    \textit{Fin de fichero (EOF)} \\
    \> G17 \> Gen\_token(eof, -) \> // Transición 0:19 \\
\end{tabbing}
\newpage
\begin{tabbing}
    \hspace{1cm}\=\hspace{1cm}\=\hspace{7cm}\=\kill
    \textit{Identificadores y palabras reservadas} \\
    \> G18 \> type := GetTokenCode(lex);*\> // Transición 1:2\\
    \> \hspace{1cm}if type = id then\\
    \> \hspace{2cm}pos := SearchST(lex);**\\
    \> \hspace{2cm}if pos = NULL then\\
    \> \hspace{3cm}pos := AddID(lex);***\\
    \> \hspace{2cm}Gen\_Token(type, pos);\\
    \> \hspace{1cm}else\\
    \> \hspace{2cm}Gen\_Token(type, -);\\
    \>* La función GetTokenCode(lex) devuelve el código de token de la palabra reservada \\
    \>con la que coincida lex, o el código de ID si no es palabra reservada\\
    \>** La función SearchST(lex) devuelve la posición del identificador lex en la tabla \\
    \>de símbolos, o NULL si no ha sido insertado aún\\
    \>*** La función AddID(lex) inserta el identificador lex en la tabla de símbolos y \\
    \>devuelve su posición\\
\end{tabbing}

\subsection{Gestión de errores}
Los casos de error posibles son todas aquellas transiciones no recogidas por nuestro AFD. En estos casos, se generará un error léxico y se informará al usuario de la existencia de un error en el código fuente, junto con el número de línea y columna en la que se ha detectado el mismo. A continuación, detallamos los errores que hemos identificado y los mensajes de error correspondientes a cada uno:\\
\begin{itemize}[left=1cm]
	\item \textbf{Caracteres no reconocidos.} En caso de que el carácter leído no coincida con ninguna transición del AFD, de manera que no pueda transitar desde el estado 0, se generará un error léxico con el siguiente mensaje: \textcolor{red}{(línea:columna) ERROR: Carácter inesperado al buscar el siguiente símbolo («carácter inesperado», U+{:04X}).}
    \item \textbf{Comentario de bloque mal abierto.} En caso de que tras encontrar una `/' no se encuentre un «*», se generará un error léxico con el siguiente mensaje: \textcolor{red}{(línea:columna) ERROR: Carácter inesperado tras «/». Se esperaba «*» para abrir un comentario de bloque}
    \item \textbf{Comentario de bloque no cerrado.} Si se detecta un comentario de bloque sin cerrar, se generará un error léxico con el siguiente mensaje: \textcolor{red}{(línea:columna) ERROR: Fin de fichero inesperado. Se esperaba «*/» para cerrar el comentario de bloque.}
	\item \textbf{Valor entero fuera de rango.} Si el valor de una constante entera supera el rango permitido que abarcan los enteros de 16 bits con signo, se generará un error léxico con el siguiente mensaje: \textcolor{red}{(línea:columna) ERROR: El valor del entero es demasiado grande (máximo 32767).}
    \item \textbf{Cadena de caracteres no cerrada.} Si se detecta una cadena de caracteres sin cerrar, se generará un error léxico con el siguiente mensaje: \textcolor{red}{(línea:columna) ERROR: Fin de fichero inesperado. Se esperaba «'» para cerrar la cadena.}
    \item \textbf{Secuencia de escape no válida.} En este caso, distinguiremos dos situaciones:
    \begin{itemize}
        \item \textbf{Carácter imprimible tras la barra invertida.} Si el carácter siguiente a la barra invertida es un carácter imprimible, pero no forma ninguna secuencia de escape válida, se generará un error léxico con el siguiente mensaje: \textcolor{red}{(línea:columna) ERROR: Error en la cadena, la secuencia de escape «\textbackslash carácter» (U+{:04X}) no es válida.}
        \item \textbf{Carácter ilegal tras la barra invertida.} Si el carácter que sigue a la barra invertida no es un carácter imprimible, se generará un error léxico con el siguiente mensaje: \textcolor{red}{(línea:columna) ERROR: Error en la cadena, carácter ilegal en la secuencia de escape (U+{:04X}).}
    \end{itemize}
    \item \textbf{Carácter no permitido en la cadena.} Si se detecta un carácter no permitido en la cadena, como un salto de línea, se generará un error léxico con el siguiente mensaje: \textcolor{red}{(línea:columna) ERROR: Error en la cadena, carácter no permitido (U+{:04X}).}
    \item \textbf{Cadena de caracteres demasiado larga.} Si la longitud de la cadena supera los 64 caracteres, se generará un error léxico con el siguiente mensaje (donde x es el número de caracteres de la cadena): \textcolor{red}{(línea:columna) ERROR: La longitud de cadena excede el límite de 64 caracteres (x caracteres).}
	\item \textbf{Operadores lógicos incorrectos.} En caso de haber un \verb!&! en lugar de \verb!&&! o un \verb!|! en lugar de \verb!||!, se generará un error léxico con el siguiente mensaje (donde op será \verb!&! o \verb!|!): \textcolor{red}{(línea:columna) ERROR: Se esperaba «op» después de «op» para formar un operador.}
\end{itemize}


Para ilustrar mejor el comportamiento del programa, en la sección 8 presentamos varios casos en los que demostramos cómo el analizador maneja la detección de errores.

\section{Analizador Sintáctico}
El \textbf{Analizador Sintáctico} es la segunda etapa del proceso de análisis y tiene como objetivo principal verificar si la secuencia de tokens generada por el Analizador Léxico cumple con las reglas de la gramática del lenguaje. Su función es comprobar que la estructura del código fuente es válida, de acuerdo con la sintaxis definida para el lenguaje en cuestión. Por ello, recibe los tokens producidos por el Analizador Léxico y construye un árbol sintáctico, que representa la jerarquía y la organización de los elementos del programa. Para llevar a cabo esta tarea utiliza las reglas de una Gramática de Contexto Libre.

\subsection{Gramática de Contexto Libre del Analizador Sintáctico}

En esta sección, describimos la Gramática de Contexto Libre (gramática de tipo 2 según la jerarquía de Chomsky) que hemos diseñado para representar la estructura sintáctica del lenguaje fuente.

\begin{tabbing}
    \hspace{1cm}\=\hspace{10cm}\=\kill
    \>1. P → FUNCTION P\\
    \>2. P → STATEMENT P\\
    \>3. P → $eof$\\
    \>4. FUNCTION → function FUNTYPE $id$ ( FUNATTRIBUTES ) \{ BODY \}\\
    \>5. FUNTYPE → $void$\\
    \>6. FUNTYPE → VARTYPE\\
    \>7. VARTYPE → $int$\\
    \>8. VARTYPE → $boolean$\\
    \>9. VARTYPE → $string$\\
    \>10. FUNATTRIBUTES → $void$\\
    \>11. FUNATTRIBUTES → VARTYPE $id$ NEXTATTRIBUTE\\
    \>12. NEXTATTRIBUTE → , VARTYPE $id$ NEXTATTRIBUTE\\
    \>13. NEXTATTRIBUTE → $\lambda$\\
    \>14. BODY → STATEMENT BODY\\
    \>15. BODY → $\lambda$\\
    \>16. STATEMENT → $if$ ( EXP1 ) ATOMSTATEMENT\\
    \>17. STATEMENT → $for$ ( FORACT ; EXP1 ; FORACT ) \{ BODY \}\\
    \>18. STATEMENT → $var$ VARTYPE $id$ ;\\
    \>19. STATEMENT → ATOMSTATEMENT\\
    \>20. ATOMSTATEMENT → $id$ IDACT ;\\
    \>21. ATOMSTATEMENT → $output$ EXP1 ;\\
    \>22. ATOMSTATEMENT → $input$ $id$ ;\\
    \>23. ATOMSTATEMENT → $return$ RETURNEXP ;\\
    \>24. IDACT → ASS EXP1\\
    \>25. IDACT → ( CALLPARAM )\\
    \>26. FORACT → $id$ ASS EXP1\\
    \>27. FORACT → $\lambda$\\
    \>28. ASS → =\\
    \>29. ASS → +=\\
    \>30. CALLPARAM → EXP1 NEXTPARAM\\
    \>31. CALLPARAM → $\lambda$\\
    \>32. NEXTPARAM → , EXP1 NEXTPARAM\\
    \>33. NEXTPARAM → $\lambda$\\
    \>34. RETURNEXP → EXP1\\
    \>35. RETURNEXP → $\lambda$\\
    \>36. EXP1 → EXP2 EXPOR\\
    \>37. EXPOR → \verb!||! EXP2 EXPOR\\
    \>38. EXPOR → $\lambda$\\
    \>39. EXP2 → EXP3 EXPAND\\
    \>40. EXPAND → \verb!&&! EXP3 EXPAND\>    50. ARITHOP → +\\
    \>41. EXPAND → $\lambda$\>    51. ARITHOP → -\\
    \>42. EXP3 → EXP4 COMP\>    52. EXPATOM → $id$ IDVAL\\
    \>43. COMP → COMPOP EXP4 COMP\>    53. EXPATOM → ( EXP1 )\\
    \>44. COMP → $\lambda$\>    54. EXPATOM → cint\\
    \>45. COMPOP → >\>    55. EXPATOM → cstr\\
    \>46. COMPOP → <\>    56. EXPATOM → true\\
    \>47. EXP4 → EXPATOM ARITH\>    57. EXPATOM → false\\
    \>48. ARITH → ARITHOP EXPATOM ARITH\>    58. IDVAL → ( CALLPARAM )\\
    \>49. ARITH → $\lambda$\>    59. IDVAL → $\lambda$\\
\end{tabbing}

\subsection{Comprobación de la condición LL(1)}

Por una parte, la gramática diseñada \textbf{no es ambigua}, ya que no existen dos producciones distintas para un mismo no terminal que generen la misma cadena de terminales y no terminales.\\

Además, \textbf{no es recursiva por la izquierda}, pues ninguna de las producciones de la misma es del tipo A → A$\alpha$ (donde A es un no terminal y $\alpha$ es una cadena de terminales y no terminales).\\

No obstante, dado que múltiples reglas de la gramática diseñada tienen dos o más producciones distintas, es necesario realizar la \textbf{comprobación de la condición LL(1)}:
\begin{center}
    $\forall$ A $\in$ N, para cada par de reglas A → $\alpha$ | $\beta$\\
    - Se cumple que FIRST($\alpha$) $\cap$ FIRST($\beta$) = $\emptyset$.\\
    - Si $\beta$ puede derivar $\lambda$, entonces FIRST($\alpha$) $\cap$ FOLLOW(A) = $\emptyset$.\\
\end{center}

De esta forma, podemos observar claramente que la gramática diseñada es LL(1):
\begin{tabbing}
    \hspace{1cm}\=\hspace{2cm}\=\kill
    P → FUNCTION P | STATEMENT P | $eof$\\
    \> FIRST(FUNCTION P) = \{function\}\\
    \> FIRST(STATEMENT P) = \{if, for, var, id, output, input, return\}\\
    \> FIRST($eof$) = \{eof\}\\
    \> \> FIRST(FUNCTION P) $\cap$ FIRST(STATEMENT P) = $\emptyset$\\
    \> \> FIRST(FUNCTION P) $\cap$ FIRST($eof$) = $\emptyset$\\
    \> \> FIRST(STATEMENT P) $\cap$ FIRST($eof$) = $\emptyset$\\
    \\
    FUNTYPE → $void$ | VARTYPE\\
    \> FIRST($void$) = \{void\}\\
    \> FIRST(VARTYPE) = \{int, boolean, string\}\\
    \> \> FIRST($void$) $\cap$ FIRST(VARTYPE) = $\emptyset$\\
    \\
    VARTYPE → $int$ | $boolean$ | $string$\\
    \> FIRST($int$) = \{int\} ; FIRST($boolean$) = \{boolean\} ; FIRST($string$) = \{string\}\\
    \> \> FIRST($int$) $\cap$ FIRST($boolean$) = $\emptyset$\\
    \> \> FIRST($int$) $\cap$ FIRST($string$) = $\emptyset$\\
    \> \> FIRST($boolean$) $\cap$ FIRST($string$) = $\emptyset$\\
    \\
    FUNATTRIBUTES → $void$ | VARTYPE $id$ NEXTATTRIBUTE\\
    \> FIRST($void$) = \{void\}\\
    \> FIRST(VARTYPE $id$ NEXTATTRIBUTE) = \{int, boolean, string\}\\
    \> \> FIRST($void$) $\cap$ FIRST(VARTYPE $id$ NEXTATTRIBUTE) = $\emptyset$\\
    \\
    NEXTATTRIBUTE → , VARTYPE $id$ NEXTATTRIBUTE | $\lambda$\\
    \> FIRST(, VARTYPE $id$ NEXTATTRIBUTE) = \{,\}\\
    \> FIRST($\lambda$) = $\{\lambda\}$\\
    \> FOLLOW(NEXTATTRIBUTE) = \{)\}\\
    \> \> FIRST(, VARTYPE $id$ NEXTATTRIBUTE) $\cap$ FIRST($\lambda$) = $\emptyset$\\
    \> \> FIRST(, VARTYPE $id$ NEXTATTRIBUTE) $\cap$ FOLLOW(NEXTATTRIBUTE) = $\emptyset$\\
    \\
    BODY → STATEMENT BODY | $\lambda$\\
    \> FIRST(STATEMENT BODY) = \{for, id, if, input, output, return, var\}\\
    \> FIRST($\lambda$) = $\{\lambda\}$\\
    \> FOLLOW(BODY) = \{\}\}\\
    \> \> FIRST(STATEMENT BODY) $\cap$ FIRST($\lambda$) = $\emptyset$\\
    \> \> FIRST(STATEMENT BODY) $\cap$ FOLLOW(BODY) = $\emptyset$\\
\end{tabbing}
Procederíamos de igual manera con el resto de reglas de la gramática con dos o más producciones asociadas, pero dado que el proceso es tedioso y repetitivo, no detallaremos con tanta profundidad en las siguientes comprobaciones:
\begin{tabbing}
    \hspace{0.5cm}\=\kill
    STATEMENT → $if$ ( EXP1 ) ATOMSTATEMENT | $for$ ( FORACT ; EXP1 ; FORACT ) \{ BODY \} |\\
    $var$ VARTYPE $id$ ; | ATOMSTATEMENT\\
    \>FIRST($if$ ( EXP1 ) ATOMSTATEMENT) $\cap$ FIRST($for$ ( FORACT ; EXP1 ; FORACT ) \{ BODY \}) = $\emptyset$\\
    \>FIRST($if$ ( EXP1 ) ATOMSTATEMENT) $\cap$ FIRST($var$ VARTYPE $id$ ;) = $\emptyset$\\
    \>FIRST($if$ ( EXP1 ) ATOMSTATEMENT) $\cap$ FIRST(ATOMSTATEMENT) = $\emptyset$\\
    \>FIRST($for$ ( FORACT ; EXP1 ; FORACT ) \{ BODY \}) $\cap$ FIRST($var$ VARTYPE $id$ ;) = $\emptyset$\\
    \>FIRST($for$ ( FORACT ; EXP1 ; FORACT ) \{ BODY \}) $\cap$ FIRST(ATOMSTATEMENT) = $\emptyset$\\
    \>FIRST($var$ VARTYPE $id$ ;) $\cap$ FIRST(ATOMSTATEMENT) = $\emptyset$\\
    \\
    ATOMSTATEMENT → $id$ IDACT ; | $output$ EXP1 ; | $input$ $id$ ; | $return$ RETURNEXP ;\\
    \>FIRST($id$ IDACT ;) $\cap$ FIRST($output$ EXP1 ;) = $\emptyset$\\
    \>FIRST($id$ IDACT ;) $\cap$ FIRST($input$ $id$ ;) = $\emptyset$\\
    \>FIRST($id$ IDACT ;) $\cap$ FIRST($return$ RETURNEXP ;) = $\emptyset$\\
    \>FIRST($output$ EXP1 ;) $\cap$ FIRST($input$ $id$ ;) = $\emptyset$\\
    \>FIRST($output$ EXP1 ;) $\cap$ FIRST($return$ RETURNEXP ;) = $\emptyset$\\
    \>FIRST($input$ $id$ ;) $\cap$ FIRST($return$ RETURNEXP ;) = $\emptyset$\\
    \\
    IDACT → ASS EXP1 | ( CALLPARAM )\\
    \>FIRST(ASS EXP1) $\cap$ FIRST( ( CALLPARAM ) ) = $\emptyset$\\
    \\
    FORACT → $id$ ASS EXP1 | $\lambda$\\
    \>FIRST($id$ ASS EXP1) $\cap$ FIRST($\lambda$) = $\emptyset$\\
    \>FIRST($id$ ASS EXP1) $\cap$ FOLLOW(FORACT) = $\emptyset$\\
    \\
    ASS → = | +=\\
    \>FIRST(=) $\cap$ FIRST(+=) = $\emptyset$\\
    \\
    CALLPARAM → EXP1 NEXTPARAM | $\lambda$\\
    \>FIRST(EXP1 NEXTPARAM) $\cap$ FIRST($\lambda$) = $\emptyset$\\
    \>FIRST(EXP1 NEXTPARAM) $\cap$ FOLLOW(CALLPARAM) = $\emptyset$\\
    \\
    NEXTPARAM → , EXP1 NEXTPARAM | $\lambda$\\
    \>FIRST(, EXP1 NEXTPARAM) $\cap$ FIRST($\lambda$) = $\emptyset$\\
    \>FIRST(, EXP1 NEXTPARAM) $\cap$ FOLLOW(NEXTPARAM) = $\emptyset$\\
    \\
    RETURNEXP → EXP1 | $\lambda$\\
    \>FIRST(EXP1) $\cap$ FIRST($\lambda$) = $\emptyset$\\
    \>FIRST(EXP1) $\cap$ FOLLOW(RETURNEXP) = $\emptyset$\\
\end{tabbing}
\begin{tabbing}
    \hspace{0.5cm}\=\hspace{10cm}\=\kill
    Repetiríamos el mismo proceso para las reglas restantes:\\
    \\
    \>EXPOR → \verb!||! EXP2 EXPOR | $\lambda$\>EXPAND → \verb!&&! EXP3 EXPAND | $\lambda$\\
    \>COMP → COMPOP EXP4 COMP | $\lambda$\>COMPOP → > | <\\
    \>ARITH → ARITHOP EXPATOM ARITH | $\lambda$\>ARITHOP → + | -\\
    \>EXPATOM → $id$ IDVAL | ( EXP ) | $cint$ | $cstr$ | $true$ | $false$\\
    \>IDVAL → ( CALLPARAM ) | $\lambda$
\end{tabbing}
Cabe destacar, además, que la comprobación de la condición LL(1) ha sido contrastada exitosamente con los resultados obtenidos por la herramienta de apoyo ofrecida en la sección de herramientas de la página web del departamento.\\

\subsection{Pseudo-código con funciones del Analizador Sintáctico}

En esta sección, presentamos el diseño en pseudo-código de las funciones que conforman el Analizador Sintáctico. Para cada símbolo no terminal de la gramática, se implementa una función que sigue un esquema de if-then-else anidado, donde cada rama corresponde a una posible regla. El token recibido desde el Analizador Léxico determina la rama que se ejecuta, iniciando el recorrido del consecuente de la regla seleccionada. Para cada símbolo del consecuente:

\begin{itemize}
    \item Si resulta ser un terminal, se equipara con el token actual. En el caso de que coincidan, se le solicita un nuevo token al Analizador Léxico y, si no, se genera un error sintáctico. 
    \item Si es un no terminal, se realiza una llamada recursiva a la función correspondiente. El main del Analizador Sintáctico inicia con la solicitud del primer token y llamando a la función del axioma. 
\end{itemize}

Si al terminar esta función la cadena se ha procesado por completo, el análisis concluye con éxito. En caso contrario, se reporta un error sintáctico. En el anexo B presentamos la estructura de dichas funciones.

\section{Analizador Semántico}
El \textbf{Analizador Semántico} es la tercera etapa del procesamiento de un lenguaje, cuyo objetivo es asegurar que el programa cumpla con las reglas y las expectativas semánticas del lenguaje. Una vez que el código ha sido analizado léxica y sintácticamente, el Analizador Semántico se encarga de verificar que las operaciones y relaciones entre los elementos del programa sean lógicas y coherentes. 

\subsection{Traducción Dirigida por la Sintaxis}
Para realizar este análisis semántico se debe llevar a cabo la \textbf{Traducción Dirigida por la Sintaxis}. La hemos representado mediante un \textbf{Esquema de Traducción}, para lo que se deben añadir las \textbf{Acciones Semánticas} directamente a la gramática del Analizador Sintáctico. 

\begin{tabbing}
    \hspace{1cm}\=\hspace{4.5cm}\=\hspace{2cm}\=\hspace{1cm}\=\hspace{1cm}\=\kill
    \>0. P' → \textbf{\{TSG := CrearTS(); TSL := NULL; DespG := 0\}} P \textbf{\{DestruirTS(TSG)\}}\\
    \>1. P → FUNCTION P\\
    \>2. P → STATEMENT P\\
    \>3. P → $eof$\\
    \>4. FUNCTION → function \textbf{\{implicitDeclaration := false\}}\\
    \>              \> FUNTYPE $id$ \textbf{\{TSL := CrearTS(); DespL := 0\}} \\
    \>              \>( FUNATTRIBUTES \textbf{\{InsertaTipoTS(id.pos, }\\
    \>              \> \textbf{FUNATTRIBUTES.tipo → FUNTYPE.tipo);}\\
    \>           \>\textbf{InsertaEtiquetaTS(id.pos, nueva\_etiqueta())\}}\\
    \>              \>) \{ BODY \}\> \textbf{\{if (FUNTYPE.tipo != BODY.tipoRet)}\\
    \>              \>          \>             \>   \textbf{then error(101)}\\
    \>              \>          \> \textbf{DestruirTS(TSL)\}}\\
    \>5. FUNTYPE → $void$ \textbf{\{FUNTYPE.tipo := void\}}\\
    \>6. FUNTYPE → VARTYPE \textbf{\{FUNTYPE.tipo := VARTYPE.tipo\}}\\
    \>7. VARTYPE → $int$ \textbf{\{VARTYPE.tipo := int; := VARTYPE.ancho := 1\}}\\
    \>8. VARTYPE → $boolean$ \textbf{\{VARTYPE.tipo := log; := VARTYPE.ancho := 1\}}\\
    \>9. VARTYPE → $string$ \textbf{\{VARTYPE.tipo := str; := VARTYPE.ancho := 64\}}\\
    \>10. FUNATTRIBUTES → $void$ \textbf{\{FUNATTRIBUTES.tipo := void\}}\\
    \>11. FUNATTRIBUTES → \>VARTYPE $id$ \textbf{\{if (BuscaTipoTS(id.pos) != NULL) then} \\
    \>              \>          \>    \>\textbf{FUNATTRIBUTES.tipo := tipo\_error;}\\
    \>              \>          \>    \>\textbf{error(000)}\\
    \>              \>          \>    \textbf{InsertaTipoTS(id.pos, VARTYPE.tipo);}\\
    \>                    \>        \>\textbf{InsertaDespTS(id.pos, DespL);}\\
    \>                    \>        \>\textbf{DespL := DespL + VARTYPE.ancho\}}\\
    \>                    \>NEXTATTRIBUTE \textbf{\{FUNATTRIBUTES.tipo :=}\\
    \>                    \>                      \> \textbf{if (NEXTATTRIBUTE.tipo != void) then}\\
    \>                    \>                      \> \>\textbf{VARTYPE.tipo × NEXTATTRIBUTE.tipo}\\
    \>                    \>                      \> \textbf{else VARTYPE.tipo\}}\\
    \>12. NEXTATTRIBUTE → , VARTYPE $id$ \textbf{\{if (BuscaTipoTS(id.pos) != NULL)then} \\
    \>              \>          \>    \>\textbf{FUNATTRIBUTES.tipo := tipo\_error;}\\
    \>              \>          \>    \>\textbf{error(000)}\\
    \>              \>          \>    \textbf{InsertaTipoTS(id.pos, VARTYPE.tipo);}\\
    \>                    \>        \>\textbf{InsertaDespTS(id.pos, DespL);}\\
    \>                    \>        \>\textbf{DespL := DespL + VARTYPE.ancho\}}\\
    \>                    \>NEXTATTRIBUTE$_1$ \textbf{\{NEXTATTRIBUTE.tipo :=}\\
    \>                    \>                      \> \textbf{if (NEXTATTRIBUTE$_1$.tipo != void) then}\\
    \>                    \>                      \> \>\textbf{VARTYPE.tipo $\times$ NEXTATTRIBUTE$_1$.tipo}\\
    \>                    \>                      \> \textbf{else VARTYPE.tipo\}}\\
    \>13. NEXTATTRIBUTE → $\lambda$ \textbf{\{NEXTATTRIBUTE.tipo := void\}}\\
    \>14. BODY → STATEMENT BODY$_1$\\
    \>                    \>\textbf{\{BODY.tipo := if (STATEMENT.tipo = tipo\_ok)}\\
    \>                    \>           \>          \>\textbf{then BODY$_1$.tipo}\\
    \>                    \>           \>\textbf{else tipo\_error}\\
    \>                    \>\textbf{BODY.tipoRet := if(STATEMENT.tipoRet = BODY$_1$.tipoRet}\\
    \>                    \>           \>\textbf{or STATEMENT.tipoRet = void) then}\\
    \>                    \>           \>          \>\textbf{BODY$_1$.tipoRet}\\
    \>                    \>           \>\textbf{else if (BODY$_1$.tipoRet = void) then}\\
    \>                    \>           \>          \>\textbf{STATEMENT.tipoRet}\\
    \>                    \>           \>\textbf{else tipo\_error\}}\\
    \>15. BODY → $\lambda$ \textbf{\{BODY.tipo := tipo\_ok; BODY.tipoRet := void\}}\\
    \>16. STATEMENT → $if$ ( EXP1 ) ATOMSTATEMENT\\
    \>                    \> \textbf{\{STATEMENT.tipo := if (EXP1.tipo != log)}\\
    \>                    \> \> \>\textbf{then tipo\_error}\\
    \>                    \> \> \textbf{else ATOMSTATEMENT.tipo}\\
    \>                    \> \textbf{STATEMENT.tipoRet := ATOMSTATEMENT.tipoRet\}}\\
    \\
    \\
    \>17. STATEMENT → $for$ ( FORACT$_1$ \textbf{\{if (FORACT$_1$.tipo != tipo\_ok) then }\\
    \>                    \>                \>      \>\textbf{error(103)\}}\\
    \>                    \> ; EXP1 \textbf{\{if (EXP1.tipo != log) then }\\
    \>                    \>                \>      \>\textbf{error(103)\}}\\
    \>                    \> ; FORACT$_2$ \textbf{\{if (FORACT$_2$.tipo != tipo\_ok) then }\\
    \>                    \>                \>      \>\textbf{error(103)\}}\\
    \>                    \> ) \{ BODY \} \textbf{\{STATEMENT.tipo := BODY.tipo}\\
    \>                    \> \>\textbf{STATEMENT.tipoRet := BODY.tipoRet\}}\\

    \>18. STATEMENT → $var$ VARTYPE $id$ ; \textbf{\{if(BuscaTipoTS(id.pos) != NULL)} \\
    \>              \>          \>
    \>\textbf{then error(000)}\\
    \>                    \> \textbf{insertaTipoTS(id.pos, VARTYPE.tipo)}\\
    \>                    \> \textbf{if(TSL != NULL) then }\\
    \>                    \> \> \textbf{insertaDespTS(id.pos,despG)}\\
    \>                    \> \> \textbf{despG := despG + VARTYPE.ancho}\\
    \>                    \> \textbf{else}\\
    \>                    \> \> \textbf{insertaDespTS(id.pos,despL)}\\
    \>                    \> \> \textbf{despL := despL + VARTYPE.ancho}\\
    \>                    \> \textbf{STATEMENT.tipo := tipo\_ok\}}\\
    \>19. STATEMENT → ATOMSTATEMENT \textbf{\{STATEMENT.tipo := ATOMSTATEMENT.tipo}\\
    \>                    \> \textbf{STATEMENT.tipoRet := ATOMSTATEMENT.tipoRet\}}\\

    \>20. ATOMSTATEMENT → $id$ IDACT ; \> \> \textbf{\{if(IDACT.funCall)\{}\\
    \> \> \> \textbf{if (id.tipo != R $\rightarrow$ T) then}\\
    \> \> \> \> \textbf{ATOMSTATEMENT.tipo := tipo\_error}\\
    \> \> \> \> \textbf{error(201)}\\
    \> \> \> \textbf{else if (IDACT.tipo = tipo\_error)}\\
    \> \> \> \> \textbf{then ATOMSTATEMENT = tipo\_error}\\
    \> \> \> \textbf{else if (IDACT.tipo = R.tipo)}\\
    \> \> \> \> \textbf{then ATOMSTATEMENT = tipo\_ok}\\
    \> \> \> \textbf{else}\\
    \> \> \> \> \textbf{ATOMSTATEMENT = tipo\_error}\\
    \> \> \> \> \textbf{error(203)}\\
    \> \> \textbf{\} else \{}\\
    \> \> \> \textbf{if (id.tipo = R $\rightarrow$ T) then}\\
    \> \> \> \> \textbf{ATOMSTATEMENT.tipo := tipo\_error}\\
    \> \> \> \> \textbf{error(202)}\\
    \> \> \> \textbf{else if (IDACT.tipo = tipo\_error)}\\
    \> \> \> \> \textbf{then ATOMSTATEMENT = tipo\_error}\\
    \> \> \> \textbf{else if (IDACT.tipo = R.tipo)}\\
    \> \> \> \> \textbf{then ATOMSTATEMENT = tipo\_ok}\\
    \> \> \> \textbf{else}\\
    \> \> \> \> \textbf{ATOMSTATEMENT = tipo\_error}\\
    \> \> \> \> \textbf{error(200)}\\
    \>21. ATOMSTATEMENT → $output$ EXP1 ; \textbf{\{if (EXP1.tipo $\in$ \{int, str\}) then}\\
    \>                    \> \> \> \textbf{ATOMSTATEMENT.tipo := tipo\_ok}\\
    \>                    \> \>\textbf{else} \\
    \> \> \> \textbf{ATOMSTATEMENT.tipo := tipo\_error}\\
    \> \> \> \textbf{error(104)}\\
    \>                    \> \textbf{ATOMSTATEMENT.tipoRet := void\}}\\
    \>22. ATOMSTATEMENT → $input$ $id$ ; \textbf{\{if (buscaTipoTS(id.pos) $\in$ \{int, str\})}\\
    \>                    \> \>\textbf{then ATOMSTATEMENT.tipo := tipo\_ok}\\
    \>                    \> \textbf{else} \\
    \> \> \> \textbf{ATOMSTATEMENT.tipo := tipo\_error}\\
    \> \> \> \textbf{error(105)}\\
    \>                    \> \textbf{ATOMSTATEMENT.tipoRet := void\}}\\
    \>23. ATOMSTATEMENT → $return$ RETURNEXP ; \textbf{\{ATOMSTATEMENT.tipo :=}\\
    \>                    \> \textbf{if (RETURNEXP.tipo != tipo\_error) then tipo\_ok}\\
    \>                    \> \textbf{else tipo\_error}\\
    \>                    \> \textbf{ATOMSTATEMENT.tipoRet := RETURNEXP.tipo\}}\\
    \\
    \>24. IDACT → ASS EXP1 \textbf{\{IDACT.tipo :=}\\
    \>                    \> \textbf{if ((ASS.sum = true AND EXP1.tipo $\in$ \{int, str\})}\\
    \>                    \> \>\textbf{OR ASS.sum = false) then EXP1.tipo}\\
    \>                    \> \textbf{else tipo\_error\}}\\
    \>25. IDACT → ( CALLPARAM ) \textbf{\{IDACT.tipo := CALLPARAM.tipo\}}\\
    \>26. FORACT → $id$ ASS EXP1 \textbf{\{if (buscaTipoTS(id.pos) != int)}\\
    \> \> \> \textbf{FORACT.tipo := tipo\_error}\\
    \> \> \> \textbf{error(102)}\\
    \> \> \textbf{else if EXP1.tipo = tipo\_error}\\
    \> \> \> \textbf{FORACT.tipo := tipo\_error}\\
    \> \> \textbf{else if EXP1.tipo != int} \\
        \> \> \> \textbf{FORACT.tipo := tipo\_error}\\
    \> \> \> \textbf{error(200)}\\
    \> \> \textbf{else FORACT.tipo := tipo\_ok}\\
    \>27. FORACT → $\lambda$ \textbf{\{FORACT.tipo := tipo\_ok\}}\\
    \>28. ASS → = \textbf{\{ASS.sum = false\}}\\
    \>29. ASS → += \textbf{\{ASS.sum = true\}}\\
    \>30. CALLPARAM → EXP1 NEXTPARAM \textbf{\{CALLPARAM.tipo :=}\\
    \>                    \> \textbf{if (EXP1.tipo = tipo\_error}\\
    \>                    \> \textbf{OR NEXTPARAM.tipo == tipo\_error) then}\\
    \>                    \> \> \textbf{tipo\_error}\\
    \>                    \> \textbf{else if (NEXTPARAM.tipo != void) then}\\
    \>                    \> \>\textbf{EXP1.tipo $\times$ NEXTPARAM.tipo}\\
    \>                    \> \textbf{else tipo\_error\}}\\ 
    \>31. CALLPARAM → $\lambda$ \textbf{\{CALLPARAM.tipo := void\}}\\
    \>32. NEXTPARAM → , EXP1 NEXTPARAM$_1$ \textbf{\{CALLPARAM.tipo :=}\\
    \>                    \> \textbf{if (EXP1.tipo = tipo\_error}\\
    \>                    \> \textbf{OR NEXTPARAM$_1$.tipo == tipo\_error) then}\\
    \>                    \> \> \textbf{tipo\_error}\\
    \>                    \> \textbf{else if (NEXTPARAM$_1$.tipo != void) then}\\
    \>                    \> \>\textbf{EXP1.tipo $\times$ NEXTPARAM$_1$.tipo}\\
    \>                    \> \textbf{else tipo\_error\}}\\ 
    \>33. NEXTPARAM → $\lambda$ \textbf{\{NEXTPARAM.tipo := void\}}\\
    \>34. RETURNEXP → EXP1 \textbf{RETURNEXP.tipo := EXP1.tipo}\\
    \>35. RETURNEXP → $\lambda$ \textbf{RETURNEXP.tipo := void}\\
    \>36. EXP1 → EXP2 EXPOR \>\textbf{\{if (EXPOR.tipo = void) then EXP1.tipo := EXP2.tipo}\\
    \>                     \> \textbf{else if (EXP2.tipo != log OR EXPOR.tipo = tipo\_error) then}\\
    \>                     \> \> \textbf{error(100)}\\
    \>                     \> \> \textbf{EXP1.tipo := tipo\_error}\\
    \>                     \> \textbf{else EXP1.tipo := log\}}\\
    \>37. EXPOR → \verb!||! EXP2 EXPOR$_1$\\
    \>                     \>\textbf{\{if (EXP2.tipo != log OR EXPOR$_1$.tipo = tipo\_error) then}\\
    \>                     \> \> \textbf{error(100)}\\
    \>                     \> \> \textbf{EXPOR.tipo := tipo\_error}\\
    \>                     \> \textbf{else EXPOR.tipo := logico\}}\\
    \>38. EXPOR → $\lambda$ \textbf{\{EXPOR.tipo := void\}}\\
    \>39. EXP2 → EXP3 EXPAND\\
    \>                     \>\textbf{\{if (EXPAND.tipo = void) then EXP2.tipo := EXP3.tipo}\\
    \>                     \> \textbf{else if (EXP3.tipo != log OR EXPAND.tipo = tipo\_error) then}\\
    \>                     \> \> \textbf{error(100)}\\
    \>                     \> \> \textbf{EXP2.tipo := tipo\_error}\\
    \>                     \> \textbf{else EXP2.tipo := log\}}\\
    \>40. EXPAND → \verb!&&! EXP3 EXPAND$_1$\\
    \>                     \>\textbf{\{if (EXP3.tipo != log OR EXPAND$_1$.tipo = tipo\_error) then}\\
    \>                     \> \> \textbf{error(100)}\\
    \>                     \> \> \textbf{EXPAND.tipo := tipo\_error}\\
    \>                     \> \textbf{else EXPAND.tipo := logico\}}\\
    \>41. EXPAND → $\lambda$ \textbf{\{EXPAND.tipo := void\}}\\
    \\
    \>42. EXP3 → EXP4 COMP \>\textbf{\{if (COMP.tipo = void) then EXP3.tipo := EXP4.tipo}\\
    \>                     \> \textbf{else if (EXP4.tipo != int OR COMP.tipo = tipo\_error) then}\\
    \>                     \> \> \textbf{error(100)}\\
    \>                     \> \> \textbf{EXP3.tipo := tipo\_error}\\
    \>                     \> \textbf{else EXP3.tipo := log\}}\\
    \>43. COMP → COMPOP EXP4 COMP$_1$\\
    \>                     \>\textbf{\{if (EXP4.tipo != int OR COMP$_1$.tipo = tipo\_error) then}\\
    \>                     \> \> \textbf{error(100)}\\
    \>                     \> \> \textbf{COMP.tipo := tipo\_error}\\
    \>                     \> \textbf{else COMP.tipo := logico\}}\\
    \>44. COMP → $\lambda$ \textbf{\{COMP.tipo := void\}}\\
    \>45. COMPOP → >\\
    \>46. COMPOP → <\\
    \>47. EXP4 → EXPATOM ARITH\\
    \>                     \>\textbf{\{if (ARITH.tipo = void) then EXP4.tipo := EXPATOM.tipo}\\
    \>                     \>\textbf{\{else if (EXPATOM.tipo != int AND EXPATOM.tipo != string) then}\\
    \> \> \> \textbf{EXP4.tipo := tipo\_error}\\
    \> \> \> \textbf{error(100)}\\
    \>                     \> \textbf{else if (ARITH.tipo = tipo\_error) then EXP4.tipo := tipo\_error}\\
    \>                     \> \textbf{else if (EXPATOM != ARITH.tipo) then}\\ 
    \> \> \> \textbf{EXP4.tipo := tipo\_error}\\
    \>                     \> \> \textbf{error(200)}\\
    \>                     \> \textbf{else EXP4.tipo := ARITH.tipo\}}\\
    \>48. ARITH → ARITHOP EXPATOM ARITH$_1$\\
    \>                     \>\textbf{\{if (ARITHOP.sum = false AND EXPATOM.tipo != int) then}\\
    \>                     \> \> \textbf{ARITH.tipo := tipo\_error}\\
    \>                     \> \> \textbf{error(100)}\\
    \> \>\textbf{else if (ARITHOP.sum = true AND EXPATOM.tipo != int}\\
    \> \> \>\textbf{AND EXPATOM.tipo != string) then}\\     
    \> \> \> \> \textbf{ARITH.tipo := tipo\_error}\\
    \> \> \> \> \textbf{error(100)}\\
    \>                     \> \textbf{else if (ARITH$_1$.tipo = tipo\_error)}\\     \> \> \> \textbf{ARITH.tipo := tipo\_error}\\
    \>                     \> \textbf{else if (ARITH$_1$.tipo != void AND EXPATOM.tipo != ARITH$_1$.tipo) then}\\ 
    \> \> \> \textbf{ARITH.tipo := tipo\_error}\\
    \>                     \> \> \textbf{error(200)}\\
    \>                     \> \textbf{else ARITH.tipo := EXPATOM.tipo\}}\\
    \>49. ARITH → $\lambda$ \textbf{\{ARITH.tipo := void\}}\\
    \>50. ARITHOP → +\\
    \>51. ARITHOP → -\\
    \>52. EXPATOM → $id$ IDVAL \textbf{\{if(IDVAL.funCall)\{}\\
    \> \> \> \textbf{if (id.tipo != R $\rightarrow$ T) then}\\
    \> \> \> \> \textbf{EXPATOM.tipo := tipo\_error}\\
    \> \> \> \> \textbf{error(201)}\\
    \> \> \> \textbf{else if (IDVAL.tipo = tipo\_error)}\\
    \> \> \> \> \textbf{then EXPATOM = tipo\_error}\\
    \> \> \> \textbf{else if (IDVAL.tipo = R.tipo)}\\
    \> \> \> \> \textbf{then EXPATOM = T.tipo}\\
    \> \> \> \textbf{else}\\
    \> \> \> \> \textbf{EXPATOM = tipo\_error}\\
    \> \> \> \> \textbf{error(203)}\\
    \> \> \textbf{\} else \{}\\
    \> \> \> \textbf{if (id.tipo = R $\rightarrow$ T) then}\\
    \> \> \> \> \textbf{EXPATOM.tipo := tipo\_error}\\
    \> \> \> \> \textbf{error(202)}\\
    \> \> \> \textbf{else}\\
    \> \> \> \> \textbf{EXPATOM = id.tipo}\\
    \>53. EXPATOM → ( EXP1 ) \>\textbf{\{EXPATOM.tipo := EXP1.tipo\}}\\
    \>54. EXPATOM → cint \>\textbf{\{EXPATOM.tipo := int\}}\\
    \>55. EXPATOM → cstr \>\textbf{\{EXPATOM.tipo := str\}}\\
    \>56. EXPATOM → true \>\textbf{\{EXPATOM.tipo := log\}}\\
    \>57. EXPATOM → false \>\textbf{\{EXPATOM.tipo := log\}}\\
    \>58. IDVAL → ( CALLPARAM )\\
    \>59. IDVAL → $\lambda$\\
\end{tabbing}


\section{Tabla de Símbolos}
La Tabla de Símbolos (TS) es una estructura de datos fundamental en la implementación de un compilador o procesador de lenguajes. Su principal objetivo es \textbf{almacenar información sobre los identificadores} (variables, funciones, etc.) que aparecen en el programa fuente y organizarla de manera eficiente para su consulta durante las fases de análisis semántico y de ejecución. En nuestro caso, hemos implementado una TS lineal y dinámica, que gestiona tanto la información de los identificadores como los alcances de los mismos.

\subsection{Descripción de su estructura}
La Tabla de Símbolos Global es esencialmente una colección de entradas, donde cada entrada corresponde a un símbolo (un identificador). Cada símbolo tiene varios atributos dependiendo de su tipo (por ejemplo, una variable, una función, un array, etc.). Sin embargo, este proyecto requiere de la existencia de Tablas Locales. Estas almacenan información sobre los identificadores dentro de un alcance específico. \\

Es por esta razón que nuestra Tabla de Símbolos tiene una estructura de pila de tablas de símbolos. Cuando se ingresa a un nuevo bloque de código (por ejemplo, al ingresar a una función), se puede crear una nueva tabla local que se apila encima de la tabla global o de las tablas locales anteriores. Cuando se sale del bloque, se elimina la tabla local, y el procesador de lenguajes vuelve a la tabla local anterior o a la global.

Nuestra estructura principal de la Tabla de Símbolos incluye:

\begin{itemize}
  \item \textbf{Struct Symbol}: Cada símbolo contiene la siguiente información:
  \begin{itemize}
    \item \textbf{Lexema}: nombre del identificador
    \item \textbf{Atributos específicos}: que pueden ser cualquier par clave-valor, como tipo, desplazamiento, número de parámetros, etc.
    \item \textbf{Tipo de identificador} (opcional): puede agregarse para especificar el tipo del identificador
  \end{itemize}

  \item \textbf{Clase SymbolTables}: La clase principal que gestiona las tablas de símbolos:
  \begin{itemize}
    \item \textbf{Clase Table}: Una tabla de símbolos que contiene:
        \begin{itemize}
            \item \textbf{Identificador de tabla}: único para cada tabla
            \item \textbf{Lista de símbolos}: vector con los símbolos de la tabla.
            \item \textbf{Mapa de nombres a posiciones}: un mapa que asocia los nombres de los símbolos con sus posiciones en el vector.
            \item \textbf{Métodos}: 
            \begin{itemize}
                \item \textbf{AddSymbol}: Agrega un nuevo símbolo a la tabla y retorna su posición.
                \item \textbf{AddAttribute}: Agrega un atributo al símbolo en la posición indicada.
                \item \textbf{SearchSymbol}: Busca un símbolo por su nombre en la tabla y devuelve su posición si lo encuentra.
                \item \textbf{WriteTable}: Escribe el contenido de la tabla por consola.
            \end{itemize}
        \end{itemize}
    \item \textbf{Contador de tablas}: para asignar un identificador único a cada tabla creada.
    \item \textbf{Lista de tablas}: un vector que contiene las tablas de símbolos en orden jerárquico.
    \item \textbf{Métodos para gestionar tablas y símbolos}: 
        \begin{itemize}          
            \item \textbf{CreateTable()}: Crea una nueva tabla de símbolos y la agrega a la lista de tablas. 
          
            \item \textbf{DestroyTable()}: Elimina la tabla de símbolos más reciente de la lista de tablas.
          
            \item \textbf{AddSymbol()}: Agrega un nuevo símbolo a la tabla más reciente, dado su nombre.
          
            \item \textbf{AddGlobalSymbol()}: Agrega un símbolo a la tabla global.
          
            \item \textbf{AddAttribute()}: Agrega un atributo a un símbolo específico en la tabla más reciente, dado su índice, el nombre del atributo y su valor.
          
            \item \textbf{SearchSymbol()}: Busca un símbolo en todas las tablas (empezando desde la tabla más reciente) y devuelve su posición si es encontrado.
          
            \item \textbf{WriteTable()}: Escribe la tabla de símbolos más reciente por consola.
        \end{itemize}
    \end{itemize}
\end{itemize}

\subsection{Organización}
En cuanto a la organización de la tabla de símbolos, se trata de un mecanismo que permite estructurar y gestionar los identificadores en un programa de manera eficiente, para que se pueda acceder a ellos correctamente.

\begin{enumerate}
    \item \textbf{Tablas Globales y Locales:}
    \begin{itemize}
        \item La \textbf{tabla global} contiene símbolos con alcance global, es decir, aquellos que pueden ser utilizados en cualquier parte del programa.
        \item Las \textbf{tablas locales} contienen símbolos con alcance local, que solo son accesibles dentro de la función o bloque donde se han declarado.
    \end{itemize}

    \item \textbf{Estructura de las Tablas:}
    \begin{itemize}
        \item Cada tabla de símbolos (global o local) puede organizarse como una lista de entradas por el nombre del símbolo.
        \item Cada entrada en la tabla contiene información relevante sobre el símbolo, como su nombre, tipo, atributos adicionales, desplazamiento en memoria, o tipo de retorno en caso de funciones.
    \end{itemize}

    \item \textbf{Organización Jerárquica:}
    \begin{itemize}
        \item Las tablas de símbolos se organizan en una \textbf{jerarquía}. Las tablas locales pueden referirse a símbolos definidos en tablas globales.
    \end{itemize}

    \item \textbf{Relación entre Tablas:}
    \begin{itemize}
        \item Las tablas locales y globales se relacionan para gestionar el alcance de los símbolos.
        \item Los símbolos locales y globales pueden compartir nombres sin interferir, ya que se encuentran en tablas separadas, lo que evita conflictos de nombres.
        \item Para \textbf{acceder a un símbolo}, el programa primero consulta la tabla local asociada al bloque o función actual. Si el símbolo no se encuentra allí, se consulta la tabla global.
    \end{itemize}

\end{enumerate}

\subsection{Ejemplo de organización de la Tabla de Símbolos}
\subsection*{1. Código de Ejemplo}
\begin{verbatim}
int x;               // Variable global
void func1() {
    int a;           // Variable local a func1
    x = a + 10;      // Uso de variable global y local
}
void func2() {
    int x;           // Variable local a func2 (oculta a la global)
    x = 20;          // Uso de variable local de func2
}
\end{verbatim}

\subsection*{2. Estructura de las Tablas de Símbolos}

\paragraph{Tabla Global:} Contiene símbolos con alcance global.

\vspace{1mm}

\begin{tabular}{|c|c|c|}
\hline
\textbf{Símbolo} & \textbf{Tipo} & \textbf{Información Adicional} \\
\hline
x       & int  & Variable global                \\
func1   & void & Función (sin retorno)          \\
func2   & void & Función (sin retorno)          \\
\hline
\end{tabular}

\paragraph{Tabla Local de func1:} Contiene los símbolos declarados en func1.

\vspace{1mm}

\begin{tabular}{|c|c|c|}
\hline
\textbf{Símbolo} & \textbf{Tipo} & \textbf{Información Adicional} \\
\hline
a       & int  & Variable local                 \\
\hline
\end{tabular}

\paragraph{Tabla Local de func2:} Contiene los símbolos declarados en func2.

\vspace{1mm}

\begin{tabular}{|c|c|c|}
\hline
\textbf{Símbolo} & \textbf{Tipo} & \textbf{Información Adicional} \\
\hline
x       & int  & Variable local (oculta a la global) \\
\hline
\end{tabular}


\section{Diseño del Gestor de Errores}

El gestor de errores permite identificar y manejar los problemas encontrados durante las fases de análisis léxico, sintáctico y semántico. Este diseño asegura que el compilador pueda ofrecer mensajes claros y detallados al usuario, mejorando la experiencia de depuración. El gestor de errores de nuestro procesador está diseñado para manejar los siguientes aspectos:

\begin{itemize}
    \item \textbf{Manejo de mensajes de error:} Cada error tiene un código único y un mensaje descriptivo que facilita su identificación y comprensión. Los mensajes están diseñados para proporcionar contexto sobre la causa del error, incluyendo información sobre el símbolo, línea y columna involucrados.
    \item \textbf{Distinción entre tipos de error:} Los errores se clasifican en léxicos, sintácticos y semánticos, cada uno con un conjunto específico de códigos y mensajes, explicados en los puntos a continuación.
    \item \textbf{Gestión del número de línea y columna:} El gestor de errores mantiene un registro del número de línea y columna mientras procesa el código fuente. Concretamente, las clases Excepción incluyen los métodos GetLine y GetColumn. Esto permite informar sobre los errores junto a su ubicación exacta en el código, lo que facilita su corrección. 
    \item \textbf{Modos de recuperación en errores léxicos y semánticos:} El sistema puede configurarse para actuar de distintas maneras ante errores: continuar procesando al omitir un carácter, una línea completa o detenerse inmediatamente si el error es crítico. Sin embargo, no hemos implementado esta opción con los errores sintácticos al ser sus errores mucho más complejos, pues pueden afectar a muchas partes del código.
\end{itemize}

\subsection*{Lista de Códigos de Error}

\subsubsection*{Errores Léxicos}
\begin{itemize}
    \item 0000 - MISSING\_COMMENT\_START: Falta el inicio del comentario de bloque.
    \item 0001 - MISSING\_COMMENT\_END: Fin de fichero inesperado; falta el cierre del comentario de bloque.
    \item 0010 - MISSING\_STRING\_END: Fin de fichero inesperado; falta el cierre de la cadena.
    \item 0020 - MISSING\_OP\_AND: Falta el operador \& para formar un operador lógico.
    \item 0021 - MISSING\_OP\_OR: Falta el operador | para formar un operador lógico.
    \item 0100 - STRING\_FORBIDDEN\_CHARACTER: Carácter no permitido dentro de la cadena.
    \item 0101 - STRING\_ESCAPE\_SEQUENCE: Secuencia de escape inválida en la cadena.
    \item 0102 - STRING\_TOO\_LONG: La cadena excede la longitud máxima permitida.
    \item 0200 - INT\_TOO\_BIG: El entero es demasiado grande para ser representado.
    \item FF00 - UNEXPECTED\_START\_CHARACTER: Carácter inesperado al buscar el siguiente símbolo.
\end{itemize}

\subsubsection*{Errores Sintácticos}
\begin{itemize}
    \item 0100 - TOP\_LEVEL\_INVALID: Elemento inesperado a nivel superior; se esperaba una declaración válida.

    \item 0200 - FUNCTION\_INVALID: Error genérico en la declaración de función.
    \item 0210 - FUNCTION\_MISSING\_IDENTIFIER: Falta el identificador de la función.
    \item 0211 - FUNCTION\_MISSING\_PAREN\_OPEN: Falta el paréntesis de apertura en la definición de atributos.
    \item 0212 - FUNCTION\_MISSING\_PAREN\_CLOSE: Falta el paréntesis de cierre en la definición de atributos.
    \item 0213 - FUNCTION\_MISSING\_BRACK\_OPEN: Falta la llave de apertura para el cuerpo de la función.
    \item 0214 - FUNCTION\_MISSING\_BRACK\_CLOSE: Falta la llave de cierre del cuerpo de la función.

    \item 0300 - FUNTYPE\_INVALID: Tipo de retorno de función no válido.

    \item 0400 - VARTYPE\_INVALID: Tipo de variable no válido.
    \item 0410 - VARTYPE\_VOID: Una variable no puede ser del tipo void.

    \item 0500 - FUNATTRIBUTES\_INVALID: Error genérico en los atributos de la función.
    \item 0510 - FUNATTRIBUTES\_EMPTY: Falta definición de atributos en la función.
    \item 0520 - FUNATTRIBUTES\_MISSING\_IDENTIFIER: Falta el nombre del atributo tras declarar su tipo.

    \item 0600 - STATEMENT\_INVALID: Sentencia inválida.
    \item 0610 - STATEMENT\_MISSING\_END\_SEMICOLON: Falta el punto y coma al final de la sentencia.
    \item 0620 - STATEMENT\_IF\_MISSING\_PAREN\_OPEN: Falta el paréntesis de apertura en la condición de un if.
    \item 0621 - STATEMENT\_IF\_MISSING\_PAREN\_CLOSE: Falta el paréntesis de cierre en la condición de un if.
    \item 0630 - STATEMENT\_FOR\_MISSING\_PAREN\_OPEN: Falta el paréntesis de apertura en la definición de un bucle for.
    \item 0631 - STATEMENT\_FOR\_MISSING\_SEMICOLON: Falta un punto y coma en la definición del bucle for.
    \item 0632 - STATEMENT\_FOR\_MISSING\_PAREN\_CLOSE: Falta el paréntesis de cierre en la definición del bucle for.
    \item 0633 - STATEMENT\_FOR\_MISSING\_BRACK\_OPEN: Falta la llave de apertura para el cuerpo del for.
    \item 0634 - STATEMENT\_FOR\_MISSING\_BRACK\_CLOSE: Falta la llave de cierre del cuerpo del for.
    \item 0640 - STATEMENT\_VAR\_MISSING\_IDENTIFIER: Falta el identificador tras el tipo de la variable.
    \item 0650 - STATEMENT\_INPUT\_MISSING\_IDENTIFIER: Falta el identificador para la entrada.

    \item 0700 - IDACT\_INVALID: Acción de identificador inválida.
    \item 0710 - IDACT\_CALL\_MISSING\_PAREN\_CLOSE: Falta el paréntesis de cierre en la llamada a la función.

    \item 0800 - ASS\_INVALID: Asignación inválida.

    \item 0900 - EXP\_INVALID: Expresión inválida.
    \item 0901 - EXP\_MISSING\_PAREN\_CLOSE: Falta el paréntesis de cierre en la expresión.

    \item 0A00 - COMP\_INVALID: Comparación inválida.

    \item 0B00 - ARITH\_INVALID: Operación aritmética inválida.
\end{itemize}


\subsubsection*{Errores Semánticos}
\begin{itemize}
    \item 0000 - IDENTIFIER\_ALREADY\_EXISTS: El identificador ya ha sido declarado en el mismo ámbito.
    \item 0100 - INVALID\_TYPE: El tipo especificado no es válido.
    \item 0101 - INVALID\_RETURN\_TYPE: El tipo de retorno no coincide con lo esperado.
    \item 0102 - INVALID\_IF\_CONDITION\_TYPE: El tipo en la condición del bloque if no es válido.
    \item 0103 - INVALID\_FOR\_ACTION\_TYPE: El tipo en la acción del bucle for no es válido.
    \item 0104 - INVALID\_FOR\_CONDITION\_TYPE: El tipo en la condición del bucle for no es válido.
    \item 0105 - INVALID\_OUTPUT\_TYPE: El tipo especificado en la salida no es válido.
    \item 0106 - INVALID\_INPUT\_TYPE: El tipo especificado en la entrada no es válido.
    \item 0200 - INCOHERENT\_TYPES: Incoherencia entre tipos.
    \item 0201 - INCOHERENT\_VARIABLE\_AS\_FUNCTION\_TYPES: La variable ha sido usada como una función con tipos incoherentes.
    \item 0202 - INCOHERENT\_FUNCTION\_AS\_VARIABLE\_TYPES: La función ha sido usada como una variable con tipos incoherentes.
    \item 0203 - INCOHERENT\_CALL\_PARAM\_TYPES: Los parámetros de la llamada no coinciden con los tipos esperados.
\end{itemize}



\section{Demostración del funcionamiento}

\newpage
d \appendix
\section{Anexo - Analizador Léxico}

\lstset{
  language=C,
  basicstyle=\ttfamily\scriptsize,
  keywordstyle=,
  showstringspaces=false,
  escapeinside={(*@}{@*)},
}

\begin{enumerate}

    \item \textbf{Caso 1:} Funcionamiento correcto
    \begin{tcolorbox}[title={Código fuente}, colback=white]
        \begin{lstlisting}
/* Declaraciones válidas */
var boolean a;
var int b;
var string d;

/* Se vuelve a declarar la variable «a», pero no es error léxico. */
var int a;

/* El tipo «bool» no existe, se trata como identificador. */
var bool err;
        \end{lstlisting}
    \end{tcolorbox}

    \begin{tcolorbox}[title={Volcado del fichero de tokens}, colback=white]
        \begin{lstlisting}
<var, >
<bool, >
<id, 0>
<scol, >
<var, >
<int, >
<id, 1>
<scol, >
<var, >
<str, >
<id, 2>
<scol, >
<var, >
<int, >
<id, 0>
<scol, >
<var, >
<id, 3>
<id, 4>
<scol, >
<eof, >
        \end{lstlisting}
    \end{tcolorbox}

    \begin{tcolorbox}[title={Volcado del fichero de la tabla de símbolos}, colback=white]
        \begin{lstlisting}
Tabla Global #0:
*'a'
*'b'
*'d'
*'bool'
*'err'
        \end{lstlisting}
    \end{tcolorbox}


    \item \textbf{Caso 2:} Funcionamiento correcto
    \begin{tcolorbox}[title={Código fuente}, colback=white]
        \begin{lstlisting}
function void println(string s) {
    output s;
    output '\n';
}

println('¡Hola mundo!');
println('Eso son llamadas a \'output\' usando una función.');
        \end{lstlisting}
    \end{tcolorbox}

    \begin{tcolorbox}[title={Volcado del fichero de tokens}, colback=white]
        \begin{lstlisting}
<fn, >
<void, >
<id, 0>
<po, >
<str, >
<id, 1>
<pc, >
<cbo, >
<out, >
<id, 1>
<scol, >
<out, >
<cstr, "\n">
<scol, >
<cbc, >
<id, 0>
<po, >
<cstr, "¡Hola mundo!">
<pc, >
<scol, >
<id, 0>
<po, >
<cstr, "Eso son llamadas a \'output\' usando una función.">
<pc, >
<scol, >
<eof, >
        \end{lstlisting}
    \end{tcolorbox}

    \begin{tcolorbox}[title={Volcado del fichero de la tabla de símbolos}, colback=white]
        \begin{lstlisting}
Tabla Global #0:
*'println'
*'s'
        \end{lstlisting}
    \end{tcolorbox}


    \item \textbf{Caso 3:} Funcionamiento correcto
    \begin{tcolorbox}[title={Código fuente}, colback=white]
        \begin{lstlisting}
/* Leemos dos números del usuario. Las variables sin declarar se suponen globales y enteras. */
input a;
input b;

/* Comparamos los números entre sí. */
if (a < b) {
    output '\'a\' es menor que \'b\'.';
}
if (a > b) {
    output '\'a\' es mayor que \'b\'.';
}

output '\n';

/* Operamos con los números. */

output 'a + b: ';
output a + b;

output 'a - b: ';
output a - b;

output '\n';
        \end{lstlisting}        
    \end{tcolorbox}

    \begin{tcolorbox}[title={Volcado del fichero de tokens}, colback=white]
        \begin{lstlisting}
<in, >
<id, 0>
<scol, >
<in, >
<id, 1>
<scol, >
<if, >
<po, >
<id, 0>
<ls, >
<id, 1>
<pc, >
<cbo, >
<out, >
<cstr, "\'a\' es menor que \'b\'.">
<scol, >
<cbc, >
<if, >
<po, >
<id, 0>
<gr, >
<id, 1>
<pc, >
<cbo, >
<out, >
<cstr, "\'a\' es mayor que \'b\'.">
<scol, >
<cbc, >
<out, >
<cstr, "\n">
<scol, >
<out, >
<cstr, "a + b: ">
<scol, >
<out, >
<id, 0>
<sum, >
<id, 1>
<scol, >
<out, >
<cstr, "a - b: ">
<scol, >
<out, >
<id, 0>
<sub, >
<id, 1>
<scol, >
<out, >
<cstr, "\n">
<scol, >
<eof, >
        \end{lstlisting}
    \end{tcolorbox}

    \begin{tcolorbox}[title={Volcado del fichero de la tabla de símbolos}, colback=white]
        \begin{lstlisting}
Tabla Global #0:
*'a'
*'b'
        \end{lstlisting}
    \end{tcolorbox}

    
    \item \textbf{Caso 4:} Funcionamiento erróneo
    \begin{tcolorbox}[title={Código fuente}, colback=white]
        \begin{lstlisting}
/* Un comentario de bloque sin cerrar es un error léxico, ya que se recibe un EOF inesperado.
        \end{lstlisting}      
    \end{tcolorbox}

    \begin{tcolorbox}[title={Errores detectados}, colback=white]
        \begin{lstlisting}
(2:1) ERROR: Fin de fichero inesperado. Se esperaba «*/» para cerrar el comentario de bloque.
        \end{lstlisting}
    \end{tcolorbox}


    \item \textbf{Caso 5:} Funcionamiento erróneo
    \begin{tcolorbox}[title={Código fuente}, colback=white]
        \begin{lstlisting}
/* Hay algunos símbolos por los que un token no puede empezar. */
$ % @ # ?

/* En especial, las variables no pueden empezar con «_». */
var int _error;
        \end{lstlisting}      
    \end{tcolorbox}

    \begin{tcolorbox}[title={Errores detectados}, colback=white]
        \begin{lstlisting}
(2:1) ERROR: Carácter inesperado al buscar el siguiente símbolo («$», U+0024).
(2:3) ERROR: Carácter inesperado al buscar el siguiente símbolo («%», U+0025).
(2:5) ERROR: Carácter inesperado al buscar el siguiente símbolo («@», U+0040).
(2:7) ERROR: Carácter inesperado al buscar el siguiente símbolo («#», U+0023).
(2:9) ERROR: Carácter inesperado al buscar el siguiente símbolo («?», U+003F).
(5:9) ERROR: Carácter inesperado al buscar el siguiente símbolo («_», U+005F).
        \end{lstlisting}
    \end{tcolorbox}

    
    \item \textbf{Caso 6:} Funcionamiento erróneo
    \begin{tcolorbox}[title={Código fuente}, colback=white]
        \begin{lstlisting}
/* Aunque el código esté malformado y no compile, se siguen buscando y generando tokens */
var string a = $ 'Esta cadena se lee correctamente';

/* Esto es útil para depurar los errores sin detenerse sólo en el primero. */
var int a = 2?;

/* En algunos casos, es imposible recuperar tokens en un estado válido y seguir procesando. */
var string s = 'Si no se temina el string, lee todo \'; $$ /**/;; Esto es parte de la cadena.
/* Aquí ya ha terminado la cadena, ya que lee un salto de línea no permitido. */

/* Como la última cadena ya finalizó por error, aquí se recupera y sigue procesando tokens. */
var string s2 = 'Esta cadena se procesa bien';
        \end{lstlisting}      
    \end{tcolorbox}

    \begin{tcolorbox}[title={Errores detectados}, colback=white]
        \begin{lstlisting}
(2:16) ERROR: Carácter inesperado al buscar el siguiente símbolo («$», U+0024).
(5:14) ERROR: Carácter inesperado al buscar el siguiente símbolo («?», U+003F).
(8:94) ERROR: Error en la cadena. Carácter no permitido (U+000D).
        \end{lstlisting}
    \end{tcolorbox}

\end{enumerate}

\newpage

\section{Anexo - Pseudocódigo}
\begin{lstlisting}[language=C, caption={Main del Analizador Sintáctico}]
Function A_Sint() {
    sig_tok := ALex();
    P;
    if sig_tok ≠ '$' then error();
}
\end{lstlisting}

\begin{lstlisting}[language=C, caption={Equipara}]
Function equipara (t){
    if sig_tok == t
    then sig_tok := ALex()
    else error ()
}
\end{lstlisting}

\begin{lstlisting}[language=C, caption={P}]
Function P() {
    if sig_tok == 'function' then {
        print(1);
        FUNCTION();
        P();
    }
    else if sig_tok ∈ {'for', 'id', 'if', 'input', 'output', 'return', 'var'} then {
        print(2);
        STATEMENT();
        P();
    }
    else if sig_tok == 'eof' then {
        print(3);
        equipara( eof );
    }
    else error();
}
\end{lstlisting}

\begin{lstlisting}[language=C, caption={FUNCTION}]
Function FUNCTION() {
    if sig_tok == 'function' then {
        print(4);
        equipara(function);
        FUNTYPE();
        equipara(id);
        equipara(();
        FUNATTRIBUTES();
        equipara());
        equipara({);
        BODY();
        equipara(});
    }
    else error();
}
\end{lstlisting}

\begin{lstlisting}[language=C, caption={FUNTYPE}]
Function FUNTYPE() {
    if sig_tok == 'void' then {
        print(5);
        equipara(void);
    }
    else if sig_tok ∈ {'boolean', 'int', 'string'} then {
        print(6);
        VARTYPE();
    }
    else error();
}
\end{lstlisting}

\begin{lstlisting}[language=C, caption={VARTYPE}]
Function VARTYPE() {
    if sig_tok == 'int' then {
        print(7);
        equipara(int);
    }
    else if sig_tok == 'boolean' then {
        print(8);
        equipara(boolean);
    }
    else if sig_tok == 'string' then {
        print(9);
        equipara(string);
    }
    else error();
}
\end{lstlisting}

\begin{lstlisting}[language=C, caption={FUNATTRIBUTES}]
Function FUNATTRIBUTES() {
    if sig_tok == 'void' then {
        print(10);
        equipara(void);
    }
    else if sig_tok ∈ {'boolean', 'int', 'string'} then {
        print(11);
        VARTYPE();
        equipara(id);
        NEXTATTRIBUTE();
    }
    else error();
}
\end{lstlisting}

\begin{lstlisting}[language=C, caption={NEXTATTRIBUTE}]
Function NEXTATTRIBUTE() {
    if sig_tok == ',' then {
        print(12);
        equipara(,);
        VARTYPE();
        equipara(id);
        NEXTATTRIBUTE();
    }
    else if sig_tok = ')' then 
        print(13);
    else error();
}
\end{lstlisting}

\begin{lstlisting}[language=C, caption={BODY}]
Function BODY() {
    if sig_tok ∈ {'for', 'id', 'if', 'input', 'output', 'return', 'var'} then {
        print(14);
        STATEMENT();
        BODY();
    }
    else if sig_tok = '}' then 
        print(15);
    else error();
}
\end{lstlisting}

\begin{lstlisting}[language=C, caption={ATOMSTATEMENT}]
Function ATOMSTATEMENT() {
    if sig_tok == 'id' then {
        print(16);
        equipara(id);
        IDACT();
        equipara( ; );
    }
    else if sig_tok == 'output' then {
        print(17);
        equipara(output);
        EXP();
        equipara( ; );
    }
    else if sig_tok == 'input' then {
        print(18);
        equipara(input);
        equipara(id);
        equipara( ; );
    }
    else if sig_tok == 'return' then {
        print(19);
        equipara(return);
        RETURNEXP();
        equipara( ; );
    }
    else error();
}
\end{lstlisting}

\begin{lstlisting}[language=C, caption={IDACT}]
Function IDACT() {
    if sig_tok ∈ {'=', '+='} then {
        print(20);
        ASS();
        EXP();
    }
    else if sig_tok == '(' then {
        print(21);
        equipara(();
        CALLPARAM();
        equipara());
    }
    else error();
}
\end{lstlisting}

\begin{lstlisting}[language=C, caption={FORACT}]
Function FORACT() {
    if sig_tok == 'id' then {
        print(22);
        equipara(id);
        ASS();
        EXP();
    }
    else if sig_tok ∈ {')', ';'} then {
        print(23);
    }
    else error();
}
\end{lstlisting}

\begin{lstlisting}[language=C, caption={ASS}]
Function IDACT() {
    if sig_tok == '=' then {
        print(24);
        equipara(=);
    }
    else if sig_tok == '+=' then {
        print(25);
        equipara(+=);
    }
    else error();
}
\end{lstlisting}

\begin{lstlisting}[language=C, caption={CALLPARAM}]
Function CALLPARAM() {
    if sig_tok ∈ {'(', 'cint', 'cstr', 'false', 'id', 'true', 'lambda'} then {
        print(26);
        EXP();
        NEXTPARAM();
    }
    else if sig_tok == ')' then {
        print(27);
    }
    else error();
}
\end{lstlisting}

\begin{lstlisting}[language=C, caption={NEXTPARAM}]
Function NEXTPARAM() {
    if sig_tok == ',' then {
        print(28);
        equipara(,);
        EXP();
        NEXTPARAM();
    }
    else if sig_tok == ')' then {
        print(29);
    }
    else error();
}
\end{lstlisting}

\begin{lstlisting}[language=C, caption={RETURNEXP}]
Function RETURNEXP() {
    if sig_tok ∈ {'(', 'cint', 'cstr', 'false', 'id', 'true'} then {
        print(30);
        EXP();
    }
    else if sig_tok == ';' then {
        print(31);
    }
    else error();
}
\end{lstlisting}

\begin{lstlisting}[language=C, caption={EXP}]
Function EXP() {
    print(32);
    A();
    EXP1();
}
\end{lstlisting}

\begin{lstlisting}[language=C, caption={EXP1}]
Function EXP1() {
    if sig_tok ∈ {'&&', '||'} then {
        print(33);
        LOGOP();
        A();
        EXP1();
    }
    else if sig_tok ∈ {')', ';', ','} then {
        print(34);
    }
    else error();
}
\end{lstlisting}

\begin{lstlisting}[language=C, caption={LOGOP}]
Function LOGOP() {
    if sig_tok == '&&' then {
        print(35);
        equipara(&&);
    }
    else if sig_tok == '||' then {
        print(36);
        equipara(||);
    }
    else error();
}
\end{lstlisting}

\begin{lstlisting}[language=C, caption={A}]
Function A() {
    print(37);
    B();
    A1();
}
\end{lstlisting}

\begin{lstlisting}[language=C, caption={A1}]
Function A1() {
    if sig_tok ∈ {'>', '<'} then {
        print(38);
        COMPOP();
        B();
        A1();
    }
    else if sig_tok ∈ {'&&', '||', ')', ';', ','} then {
        print(39);
    }
    else error();
}
\end{lstlisting}

\begin{lstlisting}[language=C, caption={COMPOP}]
Function COMPOP() {
    if sig_tok == '>' then {
        print(40);
        equipara(>);
    }
    else if sig_tok == '<' then {
        print(41);
        equipara(<);
    }
    else error();
}
\end{lstlisting}

\begin{lstlisting}[language=C, caption={B}]
Function B() {
    print(42);
    EXPATOM();
    B1();
}
\end{lstlisting}

\begin{lstlisting}[language=C, caption={B1}]
Function B1() {
    if sig_tok ∈ {'+', '-'} then {
        print(43);
        ARITHMETICOP();
        EXPATOM();
        B1();
    }
    else if sig_tok ∈ {'&&', '||', ')', ';', ',', '<', '>'} then {
        print(44);
    }
    else error();
}
\end{lstlisting}

\begin{lstlisting}[language=C, caption={ARITHMETICOP}]
Function ARITHMETICOP() {
    if sig_tok == '+' then {
        print(45);
        equipara(+);
    }
    else if sig_tok == '-' then {
        print(46);
        equipara(-);
    }
    else error();
}
\end{lstlisting}

\begin{lstlisting}[language=C, caption={EXPATOM}]
Function EXPATOM() {
    if sig_tok == 'id' then {
        print(47);
        equipara(id);
        C();
    }
    else if sig_tok == '(' then {
        print(48);
        equipara(();
        EXP();
        equipara());
    }
    else if sig_tok == 'cint' then {
        print(49);
        equipara(cint);
    }
    else if sig_tok == 'cstr' then {
        print(50);
        equipara(cstr);
    }
    else if sig_tok == 'true' then {
        print(51);
        equipara(true);
    }
    else if sig_tok == 'false' then {
        print(52);
        equipara(false);
    }
    else error();
}
\end{lstlisting}

\begin{lstlisting}[language=C, caption={C}]
Function C() {
    if sig_tok == '(' then {
        print(53);
        equipara(();
        CALLPARAM();
        equipara());
    }
    else if sig_tok ∈ {'&&', '||', ')', ';', ',', '<', '>', '+','-'} then {
        print(54);
    }
    else error();
}
\end{lstlisting}


\newpage

\section{Anexo - Analizador Sintáctico}

\begin{enumerate}

    \item \textbf{Caso 1:} Funcionamiento correcto
    \begin{tcolorbox}[title={Código fuente}, colback=white, breakable]
        \begin{lstlisting}
/* Programa con con sólo declaraciones de nivel superior. */
var boolean a;
var int b;
var string c;

/* Declarar de nuevo una variable es error semántico, no sintáctico. */
var int a;
        \end{lstlisting}
    \end{tcolorbox}

    \begin{tcolorbox}[title={Volcado del fichero de parse}, colback=white]
Des 2 18 8 2 18 7 2 18 9 2 18 7 3
    \end{tcolorbox}

    \begin{tcolorbox}[title={Árbol sintáctico generado con la herramienta VASt}, colback=white]
        \begin{lstlisting}
· P (2)
  · STATEMENT (18)
    · var
    · VARTYPE (8)
      · boolean
    · id
    · ;
  · P (2)
    · STATEMENT (18)
      · var
      · VARTYPE (7)
        · int
      · id
      · ;
    · P (2)
      · STATEMENT (18)
        · var
        · VARTYPE (9)
          · string
        · id
        · ;
      · P (2)
        · STATEMENT (18)
          · var
          · VARTYPE (7)
            · int
          · id
          · ;
        · P (3)
          · eof
        \end{lstlisting}
    \end{tcolorbox}


    \item \textbf{Caso 2:} Funcionamiento correcto
    \begin{tcolorbox}[title={Código fuente}, colback=white, breakable]
        \begin{lstlisting}
/* En este programa, definimos y llamamos a funciones. */

function void println(string s) {
    output s;
    output '\n';
}

println('¡Hola mundo!');
println('Eso son llamadas a \'output\' usando una función.');
        \end{lstlisting}
    \end{tcolorbox}

    \begin{tcolorbox}[title={Volcado del fichero de parse}, colback=white]
Des 1 4 5 11 9 13 14 19 21 36 41 46 51 58 48 43 38 14 19 21 36 41 46 54 48 43 38 15 2 19 20 25 30 36 41 46 54 48 43 38 33 2 19 20 25 30 36 41 46 54 48 43 38 33 3
    \end{tcolorbox}

    \begin{tcolorbox}[title={Árbol sintáctico generado con la herramienta VASt}, colback=white, breakable]
        \begin{lstlisting}
· P (1)
  · FUNCTION (4)
    · function
    · FUNTYPE (5)
      · void
    · id
    · (
    · FUNATTRIBUTES (11)
      · VARTYPE (9)
        · string
      · id
      · NEXTATTRIBUTE (13)
        · lambda
    · )
    · {
    · BODY (14)
      · STATEMENT (19)
        · ATOMSTATEMENT (21)
          · output
          · EXP (36)
            · A (41)
              · B (46)
                · EXPATOM (51)
                  · id
                  · C (58)
                    · lambda
                · B1 (48)
                  · lambda
              · A1 (43)
                · lambda
            · EXP1 (38)
              · lambda
          · ;
      · BODY (14)
        · STATEMENT (19)
          · ATOMSTATEMENT (21)
            · output
            · EXP (36)
              · A (41)
                · B (46)
                  · EXPATOM (54)
                    · cstr
                  · B1 (48)
                    · lambda
                · A1 (43)
                  · lambda
              · EXP1 (38)
                · lambda
            · ;
        · BODY (15)
          · lambda
    · }
  · P (2)
    · STATEMENT (19)
      · ATOMSTATEMENT (20)
        · id
        · IDACT (25)
          · (
          · CALLPARAM (30)
            · EXP (36)
              · A (41)
                · B (46)
                  · EXPATOM (54)
                    · cstr
                  · B1 (48)
                    · lambda
                · A1 (43)
                  · lambda
              · EXP1 (38)
                · lambda
            · NEXTPARAM (33)
              · lambda
          · )
        · ;
    · P (2)
      · STATEMENT (19)
        · ATOMSTATEMENT (20)
          · id
          · IDACT (25)
            · (
            · CALLPARAM (30)
              · EXP (36)
                · A (41)
                  · B (46)
                    · EXPATOM (54)
                      · cstr
                    · B1 (48)
                      · lambda
                  · A1 (43)
                    · lambda
                · EXP1 (38)
                  · lambda
              · NEXTPARAM (33)
                · lambda
            · )
          · ;
      · P (3)
        · eof
        \end{lstlisting}
    \end{tcolorbox}


    \item \textbf{Caso 3:} Funcionamiento correcto
    \begin{tcolorbox}[title={Código fuente}, colback=white, breakable]
        \begin{lstlisting}
/* Leemos dos números del usuario. Las variables sin declarar se suponen globales y enteras. */
input a;
input b;

/* Comparamos los números entre sí. */
if (a < b)
    output '\'a\' es menor que \'b\'.';
if (a > b)
    output '\'a\' es mayor que \'b\'.';

output '\n';

/* Operamos con los números. */

output 'a + b: ';
output a + b;

output 'a - b: ';
output a - b;

output '\n';
        \end{lstlisting}        
    \end{tcolorbox}

    \begin{tcolorbox}[title={Volcado del fichero de parse}, colback=white, breakable]
Des 2 19 22 2 19 22 2 16 36 41 46 51 58 48 42 45 46 51 58 48 43 38 21 36 41 46 54 48 43 38 2 16 36 41 46 51 58 48 42 44 46 51 58 48 43 38 21 36 41 46 54 48 43 38 2 19 21 36 41 46 54 48 43 38 2 19 21 36 41 46 54 48 43 38 2 19 21 36 41 46 51 58 47 49 51 58 48 43 38 2 19 21 36 41 46 54 48 43 38 2 19 21 36 41 46 51 58 47 50 51 58 48 43 38 2 19 21 36 41 46 54 48 43 38 3
    \end{tcolorbox}

    \begin{tcolorbox}[title={Árbol sintáctico generado con la herramienta VASt}, colback=white, breakable]
        \begin{lstlisting}
· P (2)
  · STATEMENT (19)
    · ATOMSTATEMENT (22)
      · input
      · id
      · ;
  · P (2)
    · STATEMENT (19)
      · ATOMSTATEMENT (22)
        · input
        · id
        · ;
    · P (2)
      · STATEMENT (16)
        · if
        · (
        · EXP (36)
          · A (41)
            · B (46)
              · EXPATOM (51)
                · id
                · C (58)
                  · lambda
              · B1 (48)
                · lambda
            · A1 (42)
              · COMPOP (45)
                ·
              · B (46)
                · EXPATOM (51)
                  · id
                  · C (58)
                    · lambda
                · B1 (48)
                  · lambda
              · A1 (43)
                · lambda
          · EXP1 (38)
            · lambda
        · )
        · ATOMSTATEMENT (21)
          · output
          · EXP (36)
            · A (41)
              · B (46)
                · EXPATOM (54)
                  · cstr
                · B1 (48)
                  · lambda
              · A1 (43)
                · lambda
            · EXP1 (38)
              · lambda
          · ;
      · P (2)
        · STATEMENT (16)
          · if
          · (
          · EXP (36)
            · A (41)
              · B (46)
                · EXPATOM (51)
                  · id
                  · C (58)
                    · lambda
                · B1 (48)
                  · lambda
              · A1 (42)
                · COMPOP (44)
                  ·
                · B (46)
                  · EXPATOM (51)
                    · id
                    · C (58)
                      · lambda
                  · B1 (48)
                    · lambda
                · A1 (43)
                  · lambda
            · EXP1 (38)
              · lambda
          · )
          · ATOMSTATEMENT (21)
            · output
            · EXP (36)
              · A (41)
                · B (46)
                  · EXPATOM (54)
                    · cstr
                  · B1 (48)
                    · lambda
                · A1 (43)
                  · lambda
              · EXP1 (38)
                · lambda
            · ;
        · P (2)
          · STATEMENT (19)
            · ATOMSTATEMENT (21)
              · output
              · EXP (36)
                · A (41)
                  · B (46)
                    · EXPATOM (54)
                      · cstr
                    · B1 (48)
                      · lambda
                  · A1 (43)
                    · lambda
                · EXP1 (38)
                  · lambda
              · ;
          · P (2)
            · STATEMENT (19)
              · ATOMSTATEMENT (21)
                · output
                · EXP (36)
                  · A (41)
                    · B (46)
                      · EXPATOM (54)
                        · cstr
                      · B1 (48)
                        · lambda
                    · A1 (43)
                      · lambda
                  · EXP1 (38)
                    · lambda
                · ;
            · P (2)
              · STATEMENT (19)
                · ATOMSTATEMENT (21)
                  · output
                  · EXP (36)
                    · A (41)
                      · B (46)
                        · EXPATOM (51)
                          · id
                          · C (58)
                            · lambda
                        · B1 (47)
                          · ARITHMETICOP (49)
                            · +
                          · EXPATOM (51)
                            · id
                            · C (58)
                              · lambda
                          · B1 (48)
                            · lambda
                      · A1 (43)
                        · lambda
                    · EXP1 (38)
                      · lambda
                  · ;
              · P (2)
                · STATEMENT (19)
                  · ATOMSTATEMENT (21)
                    · output
                    · EXP (36)
                      · A (41)
                        · B (46)
                          · EXPATOM (54)
                            · cstr
                          · B1 (48)
                            · lambda
                        · A1 (43)
                          · lambda
                      · EXP1 (38)
                        · lambda
                    · ;
                · P (2)
                  · STATEMENT (19)
                    · ATOMSTATEMENT (21)
                      · output
                      · EXP (36)
                        · A (41)
                          · B (46)
                            · EXPATOM (51)
                              · id
                              · C (58)
                                · lambda
                            · B1 (47)
                              · ARITHMETICOP (50)
                                · -
                              · EXPATOM (51)
                                · id
                                · C (58)
                                  · lambda
                              · B1 (48)
                                · lambda
                          · A1 (43)
                            · lambda
                        · EXP1 (38)
                          · lambda
                      · ;
                  · P (2)
                    · STATEMENT (19)
                      · ATOMSTATEMENT (21)
                        · output
                        · EXP (36)
                          · A (41)
                            · B (46)
                              · EXPATOM (54)
                                · cstr
                              · B1 (48)
                                · lambda
                            · A1 (43)
                              · lambda
                          · EXP1 (38)
                            · lambda
                        · ;
                    · P (3)
                      · eof
        \end{lstlisting}
    \end{tcolorbox}

    
    \item \textbf{Caso 4:} Funcionamiento erróneo
    \begin{tcolorbox}[title={Código fuente}, colback=white, breakable]
        \begin{lstlisting}
/* No podemos declarar una variable con un tipo inexistente. */
var no_type err;
        \end{lstlisting}      
    \end{tcolorbox}

    \begin{tcolorbox}[title={Errores detectados}, colback=white]
        \begin{lstlisting}
(2:12) ERROR: Tipo de variable desconocido.
        \end{lstlisting}
    \end{tcolorbox}


    \item \textbf{Caso 5:} Funcionamiento erróneo
    \begin{tcolorbox}[title={Código fuente}, colback=white, breakable]
        \begin{lstlisting}
/* Una función puede usa «void» si no devuelve nada o si no toma argumentos. */
function void empty(void) {}

/* Una variable no puede ser de tipo «void». */
var void err;
        \end{lstlisting}      
    \end{tcolorbox}

    \begin{tcolorbox}[title={Errores detectados}, colback=white]
        \begin{lstlisting}
(5:9) ERROR: Una variable no puede ser de tipo «void».
        \end{lstlisting}
    \end{tcolorbox}

    
    \item \textbf{Caso 6:} Funcionamiento erróneo
    \begin{tcolorbox}[title={Código fuente}, colback=white, breakable]
        \begin{lstlisting}
/* Usar un identificador dos veces es error semántico, no sintáctico. */
var int a;

function int a(string arg) {
    output arg
    /* El sintáctico no verifica que se devuelva un valor valido. */
};

/* Podemos realizar varias acciones sobre identificadores. */

/* Asignar un valor, si es una variable. */
a = a + 3;

/* Realizar una llamada, si es función. */
result = a();

/* No realizar ninguna acción es inválido. */
a;
        \end{lstlisting}      
    \end{tcolorbox}

    \begin{tcolorbox}[title={Errores detectados}, colback=white]
        \begin{lstlisting}
(7:2) ERROR: Expresión incorrecta: Se esperaba alguna acción sobre el identificador.
        \end{lstlisting}
    \end{tcolorbox}

\end{enumerate}

\newpage

\end{document}