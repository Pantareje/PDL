\documentclass{article}

\usepackage[margin=1in]{geometry}
\usepackage{graphicx}
\usepackage{tikz}

\usetikzlibrary{automata, positioning}


\title{\textbf{Memoria: Analizador léxico}}
\author{Carmen Toribio Pérez, 22M009\\Sergio Gil Atienza, 22M046\\María Moronta Carrión, 22M111}
\date{}


\begin{document}

\maketitle

\section{Introducción}
La primera entrega de esta práctica consiste en la implementación de un Analizador Léxico y una Tabla de Símbolos. Para ello hemos creado una Gramática Regular, con la que hemos diseñado un Autómata Finito Determinista, asociado con unas determinadas Acciones Semánticas. Por otro lado hemos organizado la Tabla de Símbolos y estudiado posibles casos de error, para poder manejar correctamente la detección y el reporte de errores léxicos. 


Hemos decidido usar C++ como lenguaje de programación porque consideramos este proyecto como una gran oportunidad para practicar con un lenguaje distinto. Además, la mayor parte de la infraestructura de compiladores está escrita en C o en C++, incluyendo el proyecto LLVM. También hemos tenido en cuenta su flexibilidad y potencia, junto a la amplia variedad de utilidades que tiene su librería estándar. Comparado con Javascript, es un lenguaje más eficiente que permite un mayor control sobre los recursos del sistema, algo crucial en proyectos que requieren optimización de bajo nivel. 

\section{Gramática}
Aquí ponemos algo blablabla. \\
Aquí ponemos GR = {todas las cosas de la \underline{gramatica} (gramática)} y definimos todo enplan serio fuap.

\section{Autómata Finito Determinista}
Aquí ponemos algo de texto tmb para que quede épico.

\input{AFD.tex}

Y aquí ponemos pues lo que sea también (nose escribir).

\section{Acciones semánticas}
Mucho texto y la transición a la que corresponde cada acción

\section{Gestión de errores}
Explicando los casos de error posibles, los códigos de error asignados a cada caso etc.

\section{Tabla de Símbolos}
Solo tiene identificadores y su \underline{posicion} (posición), de momento. "muchotexto mimimi"

\section{Funcionamiento del programa??}
Breve explicación de lo que hace nuestro programa.

\section{Carmen nini}
¿La de las tildes te la sabes? Jejéjèjê.

\end{document}
