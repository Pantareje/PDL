\documentclass{article}

\usepackage[margin=1in]{geometry}
\usepackage{indentfirst}
\usepackage{enumitem}
\usepackage{fancyhdr}
\usepackage{array}
\usepackage{hyperref}

\fancypagestyle{plain}{
    \fancyhead[L]{}
    \fancyhead[C]{}
    \fancyhead[R]{}
    \renewcommand{\headrulewidth}{0pt}
    \fancyfoot[R]{\thepage}
    \fancyfoot[C]{}
    \fancyfoot[L]{Práctica 1 - Procesadores de Lenguajes}
}

\pagestyle{fancy}
\fancyhead[L]{Grupo 7}
\fancyhead[C]{}
\fancyfoot[R]{\thepage}
\fancyfoot[C]{}
\fancyfoot[L]{Práctica 1 - Procesadores de Lenguajes}

\title{Práctica 1 - Procesadores de Lenguajes}
\author{Carmen Toribio Pérez, 22M009\\Sergio Gil Atienza, 22M046\\María Moronta Carrión, 22M111}

\begin{document}

\maketitle

\section{Introducción}

La primera entrega de esta práctica consiste en la implementación de un Analizador Léxico y una Tabla de Símbolos. Para ello, hemos seguido los siguientes pasos: 
\begin{enumerate}
    \item \textbf{Identificar los Tokens} del lenguaje fuente, teniendo en cuenta las especificaciones de nuestro grupo.
    \item Construir la \textbf{Gramática Regular} que los genera.
    \item Diseñar el \textbf{Autómata Finito Determinista} equivalente a la gramática. Hemos realizado una representación de este a través de un \textbf{diagrama de estados}.
    \item Añadir las \textbf{Acciones Semánticas}, asociadas a cada una de las transiciones del diagrama de estados.
    \item Estudiar los posibles \textbf{casos de error}, para poder manejar correctamente la detección y el reporte de errores léxicos. 
\end{enumerate}

 \vspace{0.5cm}
 
 Por otro lado, hemos comenzado con la implementación de la \textbf{Tabla de Símbolos}, concretamente con el diseño de su estructura y su organización.

 \vspace{0.5cm}
 
 Finalmente, presentamos una serie de casos de prueba que demuestran el correcto funcionamiento del Procesador, así como su capacidad para manejar errores léxicos.
 
 \vspace{0.5cm}

Como integrantes del \textbf{grupo 7}, hemos tenido que cumplir con las siguientes especificaciones: 
\begin{itemize}[left=2cm]
    \item Sentencias: Sentencia repetitiva (for)
    \item Operadores especiales: Asignación con suma (+=)
    \item Técnicas de Análisis Sintáctico: Descendente Recursivo
    \item Comentarios: Comentario de bloque (/* */)
    \item Cadenas: Con comillas simples (' ')
\end{itemize}

\vspace{0.5cm}

Hemos decidido usar C++ como lenguaje de programación porque consideramos este proyecto como una gran oportunidad para practicar con un lenguaje distinto a los que estamos acostumbrados. Además, la mayor parte de la infraestructura de compiladores está escrita en C o en C++, incluyendo el proyecto LLVM (Low Level Virtual Machine). También hemos tenido en cuenta su flexibilidad y potencia, junto a la amplia variedad de utilidades que tiene su librería estándar. En comparación con JavaScript, C++ ofrece mayor eficiencia y control sobre los recursos del sistema, algo crucial en proyectos como este (que requieren optimización de bajo nivel). 

\newpage

\section{Tokens}
El primer paso a la hora de construir un Analizador Léxico es la identificación de los tokens. Para hacer la lista nos hemos basado en la actividad práctica de la plataforma Draco. Hemos decidido utilizar el mismo formato en tablas con tal de facilitar su legibilidad.

Tokens obligatorios:

\begin{table}[h!]
    \centering
    \begin{tabular}{|l|l|l|}
        \hline
        \textbf{Elemento} & \textbf{Código de Token} & \textbf{Atributo} \\ \hline
        boolean & bool & - \\ \hline
        for & for & - \\ \hline
        function & fn & - \\ \hline
        if & if & - \\ \hline
        input & in & - \\ \hline
        int & int & - \\ \hline
        output & out & - \\ \hline
        return & ret & - \\ \hline
        string & str & - \\ \hline
        var & var & - \\ \hline
        void & void & - \\ \hline
        constante entera & cint & Número \\ \hline
        Cadena (') & cstr & Cadena ("c*") \\ \hline
        Identificador & id & Número \\ \hline
        += & cumass & - \\ \hline
        = & ass & - \\ \hline
        , & com & - \\ \hline
        ; & scol & - \\ \hline
        ( & po & - \\ \hline
        ) & pc & - \\ \hline
        \{ & cbo & - \\ \hline
        \} & cbc & - \\ \hline
    \end{tabular}
\end{table}

Tokens de operadores aritméticos, lógicos y relcionales:
\begin{table}[h!]
    \centering
    \begin{tabular}{|l|l|l|}
        \hline
        \textbf{Grupo de Opciones} & \textbf{Código de Token} & \textbf{Atributo} \\ \hline
        Grupo Operadores Aritméticos: Suma (+) & sum & - \\ \hline
        Grupo Operadores Aritméticos: Resta (-) & sub & - \\ \hline
        Grupo Operadores Lógicos: Y lógico (\&\&) & and & - \\ \hline
        Grupo Operadores Lógicos: O lógico (\texttt{||}) & or & - \\ \hline
        Grupo Operadores Relacionales: Menor (<) & ls & - \\ \hline
        Grupo Operadores Relacionales: Mayor (>) & gr & - \\ \hline
    \end{tabular}
\end{table}

Tokens opcionales:
\begin{table}[h!]
    \centering
    \begin{tabular}{|l|l|l|}
        \hline
        \textbf{Grupo de Opciones} & \textbf{Código de Token} & \textbf{Atributo} \\ \hline
        Menos Unario (-) & sub & - \\ \hline
        Más Unario (+) & sum & - \\ \hline
        false & cap & - \\ \hline
        true & nocap & - \\ \hline
        EOF & eof & - \\ \hline
    \end{tabular}
\end{table}

Por tanto, los siguientes tipos de expresiones no serán identificados como tokens: los delimitadores (como los espacios en blanco o las tabulaciones), los comentarios de bloque (/* */), los saltos de línea (\textbackslash n) o las cadenas de texto vacías (' ').

\section{Gramática Regular}
Tras identificar los tokens del lenguaje 
Aquí ponemos GR = {todas las cosas de la \underline{gramatica} (gramática)} y definimos todo enplan serio fuap.

\section{Autómata Finito Determinista}
Aquí ponemos algo de texto tmb para que quede épico.

\input{AFD.tex}

Y aquí ponemos pues lo que sea también (nose escribir).

\section{Acciones semánticas}
Mucho texto y la transición a la que corresponde cada acción

\section{Gestión de errores}
Explicando los casos de error posibles, los códigos de error asignados a cada caso etc.

\section{Tabla de Símbolos}
Solo tiene identificadores y su \underline{posicion} (posición), de momento. "muchotexto mimimi"

\section{Funcionamiento del programa??}
Breve explicación de lo que hace nuestro programa.

\section{Carmen nini}
¿La de las tildes te la sabes? Jejéjèjê.

\end{document}
