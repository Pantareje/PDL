\documentclass{article}

\usepackage[margin=1in]{geometry}
\usepackage{graphicx}
\usepackage{tikz}

\usetikzlibrary{automata, positioning}


\title{\textbf{Memoria: Analizador léxico}}
\author{Carmen Toribio Pérez, 22M009\\Sergio Gil Atienza, 22M046\\María Moronta Carrión, 22M111}
\date{}


\begin{document}

\maketitle

\section{Introducción}
Hola aquí hay que poner unpoquillo de texto. "La practica consiste de mimimimi... Hemos decidido usar C++ porque mimimimi..."

\section{Gramática}
Aquí ponemos algo blablabla. \\
Aquí ponemos GR = {todas las cosas de la \underline{gramatica} (gramática)} y definimos todo enplan serio fuap.

\section{Autómata Finito Determinista}
Aquí ponemos algo de texto tmb para que quede épico.

\input{AFD.tex}

Y aquí ponemos pues lo que sea también (nose escribir).

\section{Acciones semánticas}
Mucho texto y la transición a la que corresponde cada acción

\section{Gestión de errores}
Explicando los casos de error posibles, los códigos de error asignados a cada caso etc.

\section{Tabla de Símbolos}
Solo tiene identificadores y su \underline{posicion} (posición), de momento. "muchotexto mimimi"

\section{Funcionamiento del programa??}
Breve explicación de lo que hace nuestro programa.

\section{Carmen nini}
¿La de las tildes te la sabes? Jejéjèjê.

\end{document}