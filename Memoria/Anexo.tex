\lstset{
  language=C,
  basicstyle=\ttfamily\scriptsize,
  keywordstyle=,
  showstringspaces=false,
  escapeinside={(*@}{@*)},
}

\begin{enumerate}

    \item \textbf{Caso 1:} Funcionamiento correcto
    \begin{tcolorbox}[title={Código fuente}, colback=white]
        \begin{lstlisting}
/* Declaraciones válidas */
var boolean a;
var int b;
var string d;

/* Se vuelve a declarar la variable «a», pero no es error léxico. */
var int a;

/* El tipo «bool» no existe, se trata como identificador. */
var bool err;
        \end{lstlisting}
    \end{tcolorbox}

    \begin{tcolorbox}[title={Volcado del fichero de tokens}, colback=white]
        \begin{lstlisting}
<var, >
<bool, >
<id, 0>
<scol, >
<var, >
<int, >
<id, 1>
<scol, >
<var, >
<str, >
<id, 2>
<scol, >
<var, >
<int, >
<id, 0>
<scol, >
<var, >
<id, 3>
<id, 4>
<scol, >
<eof, >
        \end{lstlisting}
    \end{tcolorbox}

    \begin{tcolorbox}[title={Volcado del fichero de la tabla de símbolos}, colback=white]
        \begin{lstlisting}
Tabla Global #0:
*'a'
*'b'
*'d'
*'bool'
*'err'
        \end{lstlisting}
    \end{tcolorbox}


    \item \textbf{Caso 2:} Funcionamiento correcto
    \begin{tcolorbox}[title={Código fuente}, colback=white]
        \begin{lstlisting}
function void println(string s) {
    output s;
    output '\n';
}

println('¡Hola mundo!');
println('Eso son llamadas a \'output\' usando una función.');
        \end{lstlisting}
    \end{tcolorbox}

    \begin{tcolorbox}[title={Volcado del fichero de tokens}, colback=white]
        \begin{lstlisting}
<fn, >
<void, >
<id, 0>
<po, >
<str, >
<id, 1>
<pc, >
<cbo, >
<out, >
<id, 1>
<scol, >
<out, >
<cstr, "\n">
<scol, >
<cbc, >
<id, 0>
<po, >
<cstr, "¡Hola mundo!">
<pc, >
<scol, >
<id, 0>
<po, >
<cstr, "Eso son llamadas a \'output\' usando una función.">
<pc, >
<scol, >
<eof, >
        \end{lstlisting}
    \end{tcolorbox}

    \begin{tcolorbox}[title={Volcado del fichero de la tabla de símbolos}, colback=white]
        \begin{lstlisting}
Tabla Global #0:
*'println'
*'s'
        \end{lstlisting}
    \end{tcolorbox}


    \item \textbf{Caso 3:} Funcionamiento correcto
    \begin{tcolorbox}[title={Código fuente}, colback=white]
        \begin{lstlisting}
/* Leemos dos números del usuario. Las variables sin declarar se suponen globales y enteras. */
input a;
input b;

/* Comparamos los números entre sí. */
if (a < b) {
    output '\'a\' es menor que \'b\'.';
}
if (a > b) {
    output '\'a\' es mayor que \'b\'.';
}

output '\n';

/* Operamos con los números. */

output 'a + b: ';
output a + b;

output 'a - b: ';
output a - b;

output '\n';
        \end{lstlisting}        
    \end{tcolorbox}

    \begin{tcolorbox}[title={Volcado del fichero de tokens}, colback=white]
        \begin{lstlisting}
<in, >
<id, 0>
<scol, >
<in, >
<id, 1>
<scol, >
<if, >
<po, >
<id, 0>
<ls, >
<id, 1>
<pc, >
<cbo, >
<out, >
<cstr, "\'a\' es menor que \'b\'.">
<scol, >
<cbc, >
<if, >
<po, >
<id, 0>
<gr, >
<id, 1>
<pc, >
<cbo, >
<out, >
<cstr, "\'a\' es mayor que \'b\'.">
<scol, >
<cbc, >
<out, >
<cstr, "\n">
<scol, >
<out, >
<cstr, "a + b: ">
<scol, >
<out, >
<id, 0>
<sum, >
<id, 1>
<scol, >
<out, >
<cstr, "a - b: ">
<scol, >
<out, >
<id, 0>
<sub, >
<id, 1>
<scol, >
<out, >
<cstr, "\n">
<scol, >
<eof, >
        \end{lstlisting}
    \end{tcolorbox}

    \begin{tcolorbox}[title={Volcado del fichero de la tabla de símbolos}, colback=white]
        \begin{lstlisting}
Tabla Global #0:
*'a'
*'b'
        \end{lstlisting}
    \end{tcolorbox}

    
    \item \textbf{Caso 4:} Funcionamiento erróneo
    \begin{tcolorbox}[title={Código fuente}, colback=white]
        \begin{lstlisting}
/* Un comentario de bloque sin cerrar es un error léxico, ya que se recibe un EOF inesperado.
        \end{lstlisting}      
    \end{tcolorbox}

    \begin{tcolorbox}[title={Errores detectados}, colback=white]
        \begin{lstlisting}
(2:1) ERROR: Fin de fichero inesperado. Se esperaba «*/» para cerrar el comentario de bloque.
        \end{lstlisting}
    \end{tcolorbox}


    \item \textbf{Caso 5:} Funcionamiento erróneo
    \begin{tcolorbox}[title={Código fuente}, colback=white]
        \begin{lstlisting}
/* Hay algunos símbolos por los que un token no puede empezar. */
$ % @ # ?

/* En especial, las variables no pueden empezar con «_». */
var int _error;
        \end{lstlisting}      
    \end{tcolorbox}

    \begin{tcolorbox}[title={Errores detectados}, colback=white]
        \begin{lstlisting}
(2:1) ERROR: Carácter inesperado al buscar el siguiente símbolo («$», U+0024).
(2:3) ERROR: Carácter inesperado al buscar el siguiente símbolo («%», U+0025).
(2:5) ERROR: Carácter inesperado al buscar el siguiente símbolo («@», U+0040).
(2:7) ERROR: Carácter inesperado al buscar el siguiente símbolo («#», U+0023).
(2:9) ERROR: Carácter inesperado al buscar el siguiente símbolo («?», U+003F).
(5:9) ERROR: Carácter inesperado al buscar el siguiente símbolo («_», U+005F).
        \end{lstlisting}
    \end{tcolorbox}

    
    \item \textbf{Caso 6:} Funcionamiento erróneo
    \begin{tcolorbox}[title={Código fuente}, colback=white]
        \begin{lstlisting}
/* Aunque el código esté malformado y no compile, se siguen buscando y generando tokens */
var string a = $ 'Esta cadena se lee correctamente';

/* Esto es útil para depurar los errores sin detenerse sólo en el primero. */
var int a = 2?;

/* En algunos casos, es imposible recuperar tokens en un estado válido y seguir procesando. */
var string s = 'Si no se temina el string, lee todo \'; $$ /**/;; Esto es parte de la cadena.
/* Aquí ya ha terminado la cadena, ya que lee un salto de línea no permitido. */

/* Como la última cadena ya finalizó por error, aquí se recupera y sigue procesando tokens. */
var string s2 = 'Esta cadena se procesa bien';
        \end{lstlisting}      
    \end{tcolorbox}

    \begin{tcolorbox}[title={Errores detectados}, colback=white]
        \begin{lstlisting}
(2:16) ERROR: Carácter inesperado al buscar el siguiente símbolo («$», U+0024).
(5:14) ERROR: Carácter inesperado al buscar el siguiente símbolo («?», U+003F).
(8:94) ERROR: Error en la cadena. Carácter no permitido (U+000D).
        \end{lstlisting}
    \end{tcolorbox}

\end{enumerate}