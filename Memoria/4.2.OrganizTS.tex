En cuanto a la organización de la tabla de símbolos, se trata de un mecanismo que permite estructurar y gestionar los identificadores en un programa de manera eficiente, para que se pueda acceder a ellos correctamente.

\begin{enumerate}
    \item \textbf{Tablas Globales y Locales:}
    \begin{itemize}
        \item La \textbf{tabla global} contiene símbolos con alcance global, es decir, aquellos que pueden ser utilizados en cualquier parte del programa.
        \item Las \textbf{tablas locales} contienen símbolos con alcance local, que solo son accesibles dentro de la función o bloque donde se han declarado.
    \end{itemize}

    \item \textbf{Estructura de las Tablas:}
    \begin{itemize}
        \item Cada tabla de símbolos (global o local) puede organizarse como una lista de entradas por el nombre del símbolo.
        \item Cada entrada en la tabla contiene información relevante sobre el símbolo, como su nombre, tipo, atributos adicionales, desplazamiento en memoria, o tipo de retorno en caso de funciones.
    \end{itemize}

    \item \textbf{Organización Jerárquica:}
    \begin{itemize}
        \item Las tablas de símbolos se organizan en una \textbf{jerarquía}. Las tablas locales pueden referirse a símbolos definidos en tablas globales.
    \end{itemize}

    \item \textbf{Relación entre Tablas:}
    \begin{itemize}
        \item Las tablas locales y globales se relacionan para gestionar el alcance de los símbolos.
        \item Los símbolos locales y globales pueden compartir nombres sin interferir, ya que se encuentran en tablas separadas, lo que evita conflictos de nombres.
        \item Para \textbf{acceder a un símbolo}, el programa primero consulta la tabla local asociada al bloque o función actual. Si el símbolo no se encuentra allí, se consulta la tabla global.
    \end{itemize}

\end{enumerate}