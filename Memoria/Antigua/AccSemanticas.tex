\begin{tabbing}
    \hspace{1cm}\=\hspace{5cm}\=\kill
    \textbf{Leer:} \\
    \> car := leer(); \> // Todas las transiciones salvo 1:2, 3:4, 6:5, 8:9 con un carácter\\
    \> \hspace{5cm}// distinto de '=', y 0:19\\
    \> car := leerCesc(); \> // Transición 6:5. Leído un caracter de escape, devuelve el carácter\\
    \> \hspace{5cm}// correspondiente (por ejemplo un 'n' devuelve un eol)
    \\
    \textbf{Concatenar:} \\
    \> lex := car; \> // Transición 0:1 \\
    \> lex := $\emptyset$; \> // Transición 0:5 \\
    \> lex := lex $\oplus$  car; \> // Transiciones 1:1, 5:5 y 6:5\\
    \\
    \textbf{Contador:} \\
    \> // Utilizado para contar el número de caracteres de la cadena \\
    \> cont := 0; \> // Transición 0:5. Inicializa el contador a 0\\
    \> cont := cont + 1; \> // Transiciones 5:5 y 6:5 \\
    \\
    \textbf{Calcular valor entero:} \\
    \> // La función val(car) devuelve el valor entero del dígito correspondiente al carácter car\\
    \> num := val(car); \> // Transición 0:3 \\
    \> num := num * 10 + val(car); \> // Transición 3:3 \\
    \\
    \hspace{1cm}\=\hspace{1cm}\=\hspace{7cm}\=\kill %cambio la indentación porque ahora habrá más tabulaciones que hacer
    \textbf{Generar token:} \\
    \textit{Cadenas y enteros} \\
    \> G1 \> if (cont $>$ 64) \> // Transición 5:7 \\
    \> \hspace{1.5cm} then error (COD\_ERROR\_STRLEN, lex) \\
    \> \hspace{1cm}else Gen\_token(cstr, lex) \\
    \> G2 \> if (num $>$ 32767) \> // Transición 3:4 \\
    \> \hspace{1.5cm} then error (COD\_ERROR\_MAXINT, num) \\
    \> \hspace{1cm}else Gen\_token(cint, num) \\
    \textit{Delimitadores y operadores de control} \\
    \> G3 \> Gen\_token(cumass, -) \> // Transición 8:9 con el carácter '='\\
    \> G4 \> Gen\_token(ass, -) \> // Transición 0:10 con el carácter '='\\
    \> G5 \> Gen\_token(com, -) \> // Transición 0:18 con el carácter ','\\
    \> G6 \> Gen\_token(scol, -) \> // Transición 0:18 con el carácter ';'\\
    \> G7 \> Gen\_token(po, -) \> // Transición 0:18 con el carácter '('\\
    \> G8 \> Gen\_token(pc, -) \> // Transición 0:18 con el carácter ')'\\
    \> G9 \> Gen\_token(cbo, -) \> // Transición 0:18 con el carácter '\{'\\
    \> G10 \> Gen\_token(cbc, -) \> // Transición 0:18 con el carácter '\}' \\
    \textit{Operadores Aritméticos, Lógicos y Relacionales} \\
    \> G11 \> Gen\_token(sum, -) \> // Transición 8:9 con un carácter distinto de '='\\
    \> G12 \> Gen\_token(sub, -) \> // Transición 0:10 con el carácter '-'\\
    \> G13 \> Gen\_token(and, -) \> // Transición 11:12 \\
    \> G14 \> Gen\_token(or, -) \> // Transición 13:14 \\
    \> G15 \> Gen\_token(ls, -) \> // Transición 0:10 con el carácter '$<$'\\
    \> G16 \> Gen\_token(gr, -) \> // Transición 0:10 con el carácter '$>$'\\
    \textit{Fin de fichero (EOF)} \\
    \> G17 \> Gen\_token(eof, -) \> // Transición 0:19 \\
\end{tabbing}
\newpage
\begin{tabbing}
    \hspace{1cm}\=\hspace{1cm}\=\hspace{7cm}\=\kill
    \textit{Identificadores y palabras reservadas} \\
    \> G18 \> type := GetTokenCode(lex);*\> // Transición 1:2\\
    \> \hspace{1cm}if type = id then\\
    \> \hspace{2cm}pos := SearchST(lex);**\\
    \> \hspace{2cm}if pos = NULL then\\
    \> \hspace{3cm}pos := AddID(lex);***\\
    \> \hspace{2cm}Gen\_Token(type, pos);\\
    \> \hspace{1cm}else\\
    \> \hspace{2cm}Gen\_Token(type, -);\\
    \>* La función GetTokenCode(lex) devuelve el código de token de la palabra reservada \\
    \>con la que coincida lex, o el código de ID si no es palabra reservada\\
    \>** La función SearchST(lex) devuelve la posición del identificador lex en la tabla \\
    \>de símbolos, o NULL si no ha sido insertado aún\\
    \>*** La función AddID(lex) inserta el identificador lex en la tabla de símbolos y \\
    \>devuelve su posición\\
\end{tabbing}